\chapter{Conclusão}
\label{consideracoes-finais}

Durante este trabalho de conclusão de curso, estudamos o desenvolvimento de software empírico e suas características, focando nos métodos ágeis de desenvolvimento de software e nas práticas de software livre. Assim como a forma que esse desenvolvimento está ligado com práticas de BDD e como as práticas (testes de aceitação) podem ser aplicadas ou não em um ambiente real de desenvolvimento que já se encontra estabelecido, além da aplicação de práticas de melhorias de usabilidade, como avaliações de heurísticas e aplicações de \textit{checklists}, através dos cenários de testes automáticos.

Os objetivos deste estudo de caso foram:
\begin{enumerate}
\item Integrar usabilidade no ciclo de vida de desenvolvimento empírico de software;
\item Identificar quais técnicas de usabilidade  podem ser utilizadas em cada fase do desenvolvimento empírico.
\item Verificar a relação de práticas de BDD e práticas de usabilidade no desenvolvimento empírico de software.
\end{enumerate}

O estudo de caso deste trabalho foi o projeto de desenvolvimento do Portal do Software Público Brasileiro, em que focamos para este trabalho no estudo das práticas de desenvolvimento dos cenários de testes de aceitação, prototipação, avaliação de heurísticas de usabilidade, aplicação de \textit{checklists} de usabilidade; assim como a relação entre essas práticas.

Para atingir dos objetivos do estudo de caso, buscamos responder as seguintes questões:
\begin{enumerate}
\item Como inserir os princípios de usabilidade no desenvolvimento empírico de software?
\item Como as práticas do BDD podem se relacionar com as pŕaticas de usabilidade?
\end{enumerate}

A partir do estudo de caso inserimos algumas práticas de usabilidade dentro do processo de desenvolvimento do Portal do Software Público Brasileiro, sendo estas práticas: prototipação, avaliação de heurísticas de usabilidade, aplicação de \textit{checklists} de usabilidade. Porém existe uma certa dificuldade em aderir tanto práticas de testes, quanto práticas de usabilidade pela equipe de desenvolvimento, principalmente durante sprints de final de \textit{release}, onde o prazo de entrega encontra-se mais perto, e a equipe tende a se preocupar mais com o desenvolvimento  da funcionalidade em si. Quando essas práticas são utilizadas de forma frequente e como planejadas, verificamos que o desenvolvimento dos testes de aceitação , assim como o uso de protótipos, pode agilizar o processo de avaliações de usabilidade por meio de \textit{checklists} e heurísticas, pois não existe a necessidade de uma funcionalidade está totalmente desenvolvida. Também verificamos que os testes de aceitação podem ser usados para verificar a aplicação das melhorias estabelecidas através das avaliações de heurísticas de usabilidade. 

Entretanto é necessário dar continuidade as práticas de usabilidade durante todo o processo de desenvolvimento, para que a usabilidade possa evoluir juntamente com o processo de desenvolvimento, assim é as atividades de usabilidade estar sempre à frente das atividades de desenvolvimento, pois necessitam de planejamento de avaliações de usabilidade e da produção de de cenários e protótipos.


%Durante este estudo de caso, formulamos a seguinte hipótese:

%\textbf{Hipótese: } A integração de práticas de usabilidade a partir de práticas do BDD resulta em melhores avaliações de usabilidade durante desenvolvimento empírico de software.

De acordo com a análise dos resultados coletados na subseção \ref{analise} deste estudo, verificamos que a aplicação de práticas de usabilidade como prototipação, avaliação de heurísticas e \textit{checklists} podem ser inseridas no ciclo de desenvolvimento partir dos cenários dos testes de aceitação. Podemos ver também nas figuras \ref{avaliacaouser}, \ref{avaliacaoinstitucion} e \ref{avaliacaosoftware} que a partir das avaliações de heurísticas nos protótipos e cenários diminuiu ou se manteve o numero de casos com problemas de criticidade alta, média ou baixa, apresentando assim indícios que o desenvolvimento empírico e suas práticas combinadas com as práticas de usabilidade, pode trazer grandes ganhos em usabilidade.


\section{Trabalhos Futuros}

Encerramos este estudo antes da entrega oficial do Portal do SPB, assim a atividade de testes de usabilidade com usuários ainda será executada, o que é um passo importante para a evolução da usabilidade do Portal e continuação do trabalho realizado neste estudo.


Nas demais sprints do projeto do Portal do Software Público, outras práticas de usabilidade podem ser utilizadas pela equipe do projeto. É importante compreender as necessidades reais dos usuários, e para isso deve consultá-los para obter um melhor feedback. Um protocolo para a realização de testes de usabilidade estão relatados no apêndice \ref{teste_u}.
%
Esses testes devem ser executados nas próximas releases e sprints do projeto e tem como objetivo avaliar o desempenho do usuário na execução de uma tarefa específica no escopo do sistema. O desempenho pode ser medido através do número de erros e tempo de execução da tarefa. 

Questionários de satisfação, como descritos no apêndice \ref{quest-satisf} também podem ser utilizados para medir de forma subjetiva o grau de satisfação do usuário, verificando assim se o sistema satisfaz as suas necessidades.

Através da aplicação do teste de usabilidade, podemos fazer uma relação com os testes de aceitação. Os testes de usabilidade podem ser inseridos em qualquer parte do ciclo de desenvolvimento fornecendo informações verdadeiras dos usuários desde o início do projeto.

Como foi aplicado um questionário no inicio do projeto para analisar o perfil de uso da versão antiga do Portal do Software Público, pode-se também ser aplicado um novo questionário para o novo portal.
%
Além disso, iniciamos uma pesquisa para compreender quais técnicas e práticas de usabilidade os profissionais que trabalham com desenvolvimento empírico utilizam. Um estudo mais detalhado poderá ser feito para uma melhor compreensão do assunto aplicado nesse TCC. 

Este trabalho também pode servir de base para um estudo mais aprofundado sobre práticas de BDD e usabilidade, além da integração dessas práticas, também pode-se analisar como uma influencia a outra.
%TO DO: DIVIDIR EM RESULTADOS, DISCUSSOES E TRABALHOS FUTUROS