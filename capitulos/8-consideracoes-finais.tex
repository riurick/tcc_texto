\chapter{Conclusão}
\label{consideracoes-finais}

Durante este trabalho de conclusão de curso, analisamos como inserir práticas de melhorias de usabilidade nos métodos empíricos de desenvolvimento, a partir das práticas de BDD.

Os objetivos deste estudo de caso foram:
\begin{enumerate}
\item Integrar usabilidade no ciclo de vida de desenvolvimento empírico de software;
\item Identificar quais técnicas de usabilidade  podem ser utilizadas em cada fase do desenvolvimento empírico.
\item Verificar a relação de práticas de BDD e práticas de usabilidade no desenvolvimento empírico de software.
\end{enumerate}

O estudo de caso deste trabalho foi o projeto de desenvolvimento do Portal do Software Público Brasileiro, em que focamos para este trabalho no estudo das práticas de desenvolvimento dos cenários de testes de aceitação, prototipação, avaliação de heurísticas de usabilidade, aplicação de \textit{checklists} de usabilidade; assim como a relação entre essas práticas.

Para atingir dos objetivos do estudo de caso, buscamos responder as seguintes questões:
\begin{enumerate}
\item Como inserir os princípios de usabilidade no desenvolvimento empírico de software?
\item Como as práticas do BDD podem se relacionar com as práticas de usabilidade?
\end{enumerate}

A partir do estudo de caso inserimos algumas práticas de usabilidade dentro do processo de desenvolvimento do Portal do 
SPB, sendo estas práticas: prototipação, avaliação de heurísticas de usabilidade, aplicação de \textit{checklists} de usabilidade. Porém existe uma certa dificuldade em aderir tanto práticas de testes, quanto práticas de usabilidade pela equipe, principalmente durante sprints de final de \textit{release}, quando o prazo de entrega encontra-se mais perto e a equipe tende a se preocupar mais com o desenvolvimento  da funcionalidade em si. Quando as práticas  de usabilidade e de testes são utilizadas de forma frequente como planejadas verificamos que os testes de aceitação podem ser usados para verificar a aplicação das melhorias estabelecidas através das avaliações de heurísticas de usabilidade. 

Entretanto é necessário dar continuidade as práticas de usabilidade durante todo o processo de desenvolvimento, para que a usabilidade possa evoluir juntamente com o processo de desenvolvimento, assim é as atividades de usabilidade estar sempre à frente das atividades de desenvolvimento, pois necessitam de planejamento de avaliações de usabilidade e da produção de de cenários e protótipos.

De acordo com a análise dos resultados coletados na subseção \ref{analise} deste estudo, verificamos que a aplicação de práticas de usabilidade como prototipação, avaliação de heurísticas e \textit{checklists} podem ser inseridas no ciclo de desenvolvimento a partir dos cenários dos testes de aceitação. Podemos ver também nas figuras \ref{avaliacaouser}, \ref{avaliacaoinstitucion} e \ref{avaliacaosoftware} que o numero de casos com problemas de usabilidade diminuiu a partir das avaliações de heurísticas dos protótipos e cenários de uso diminuiu. Assim o estudo apresentou indícios que o desenvolvimento empírico e suas práticas combinadas com as práticas de usabilidade, pode trazer grandes ganhos em usabilidade.

\section{Trabalhos Futuros}

Encerramos este estudo antes da entrega oficial do Portal do SPB, assim a atividade de testes de usabilidade com usuários ainda será executada, o que é um passo importante para a evolução da usabilidade do Portal e continuação do trabalho realizado neste estudo.

Nas demais sprints do projeto do Portal do SPB, outras práticas de usabilidade podem ser utilizadas pela equipe do projeto. É importante compreender as necessidades reais dos usuários. Um protocolo para a realização de testes de usabilidade estão relatados no apêndice \ref{teste_u}.
%
Esses testes devem ser executados nas próximas com o objetivo avaliar o desempenho do usuário na execução de uma tarefa do sistema.
%
Questionários de satisfação, descritos no apêndice \ref{quest-satisf} também podem ser utilizados para medir de forma subjetiva o grau de satisfação do usuário.

Como foi aplicado um questionário no início do projeto para analisar o perfil de uso da versão antiga do Portal do SPB, pode-se também ser aplicado um novo questionário para o novo portal.
%
Além disso, iniciamos uma pesquisa para compreender quais práticas de usabilidade os profissionais que trabalham com métodos empíricos utilizam. Um estudo mais detalhado poderá ser feito para uma melhor compreensão do assunto aplicado nesse TCC. 

Este trabalho também pode servir de base para um estudo mais aprofundado sobre práticas de BDD e usabilidade, além da integração dessas práticas, também pode-se analisar como uma influencia a outra.
