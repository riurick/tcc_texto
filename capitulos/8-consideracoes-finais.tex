\chapter{Conclusão}
\label{consideracoes-finais}

Durante este trabalho de conclusão de curso, estudamos o desenvolvimento de software empírico e suas características, focando nos métodos ágeis de desenvolvimento de software e nas práticas de software livre. Assim como a forma que esse desenvolvimento está ligado com práticas de BDD e como as práticas (testes de aceitação) podem ser aplicadas ou não em um ambiente real de desenvolvimento que já se encontra estabelecido, além da aplicação de práticas de melhorias de usabilidade, como avaliações de heurísticas e aplicações de \textit{checklists}, através dos cenários de testes automáticos.

O objetivos deste estudo de caso foram:
\begin{enumerate}
\item Verificar aplicação das práticas do BDD no processo de desenvolvimento empírico de software empírico;
\item Integrar usabilidade no ciclo de vida de desenvolvimento empírico de software;
\item Identificar quais técnicas de usabilidade  podem ser utilizadas em cada fase do desenvolvimento empírico.
\item Verificar a relação de práticas de BDD e práticas de usabilidade no desenvolvimento empírico de software.
\end{enumerate}

O estudo de caso deste trabalho foi o projeto de desenvolvimento do Portal do Software Público Brasileiro, em que focamos para este trabalho no estudo das práticas de desenvolvimento dos cenários de testes de aceitação, prototipação, avaliação de heurísticas de usabilidade, aplicação de \textit{checklists} de usabilidade; assim como a relação entre essas práticas.

A partir do estudo de caso, verificamos que as práticas de BDD são utilizadas constantemente durante o desenvolvimento empírico, porém existe uma certa dificuldade em aderir tanto práticas de testes, quanto práticas de usabilidade pela equipe de desenvolvimento, principalmente durante sprints de final de \textit{release}, onde o prazo de entrega encontra-se mais perto, e a equipe tende a se preocupar mais com o desenvolvimento da funcionalidade em si. Porém quando essas práticas são utilizadas de forma frequente e como planejadas, verificamos que o desenvolvimento dos testes de aceitação , assim como o uso de protótipos, pode agilizar o processo de avaliações de usabilidade por meio de \textit{checklists} e heurísticas, pois não existe a necessidade de uma funcionalidade está totalmente desenvolvida. 

Entretanto é necessário dar continuidade as práticas de usabilidade durante todo o processo de desenvolvimento, para que a usabilidade possa evoluir juntamente com o processo de desenvolvimento. Também verificamos que a equipe de design deve está uma sprint à frente da equipe de desenvolvimento, pois necessitam de mais tempo para o planejamento de avaliações de usabilidade e da produção de protótipos.



%Assim, verificamos que as práticas de BDD são utilizadas constamente durante o desenvolvimento empírico, porém essas práticas podem ser comprometidas quando a equipe de desenvolvimento é pressionada a focar nas funcionalidades do sistema pelo prazo da entrega. Também conseguimos integrar algumas práticas de usabilidade durante as fases do desenvolvimento, como utilização de protótipos, aplicação de \textit{checklists} e avaliações de heurísticas.

Durante este estudo de caso, formulamos a seguinte hipótese:

\textbf{Hipótese: } A integração de práticas de usabilidade a partir de práticas do BDD resulta em melhores avaliações de usabilidade durante desenvolvimento empírico de software.

De acordo com a análise dos resultados coletados na subseção \ref{analise} deste estudo, verificamos que a aplicação de práticas de usabilidade como prototipação, avaliação de heurísticas e \textit{checklists} a partir dos cenários dos testes de aceitação apresentaram evolução na usabilidade do Portal do Software Público.


O estudo apresentou resultados significantes em relação ao desenvolvimento empírico	e suas práticas combinadas com as práticas de usabilidade, mostrando que a inserção de algumas práticas dentro do processo de desenvolvimento empírico pode trazer grandes ganhos em usabilidade.

\section{Trabalhos Futuros}

Encerramos este estudo antes da entrega oficial do Portal do SPB, assim a atividade de testes de usabilidade com usuários ainda será executada, o que é um passo importante para a evolução da usabilidade do Portal e continuação do trabalho realizado neste estudo.

Além disso, uma pesquisa foi iniciada para compreender quais técnicas e práticas de usabilidade os profissionais que trabalham com software livre e métodos ágeis utilizam. Também servindo para entender a percepção dos profissionais pelas diversas técnicas existentes. Um estudo mais detalhado poderá ser feito para uma melhor compreensão do assunto aplicado nesse TCC. 

%TO DO: DIVIDIR EM RESULTADOS, DISCUSSOES E TRABALHOS FUTUROS