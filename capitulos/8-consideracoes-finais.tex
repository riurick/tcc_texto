\chapter{Conclusão}
\label{consideracoes-finais}

Durante este trabalho de conclusão de curso, estudamos o desenvolvimento de software empírico e suas características, focando nos métodos ágeis de desenvolvimento de software e nas práticas de software livre. Assim como a forma que esse desenvolvimento está ligado com práticas de BDD e como as práticas (testes de aceitação) podem ser aplicadas ou não em um ambiente real de desenvolvimento que já se encontra estabelecido, além da aplicação de práticas de melhorias de usabilidade, como avaliações de heurísticas e aplicações de \textit{checklists}, através dos cenários de testes automáticos.

O estudo de caso deste trabalho foi o projeto de desenvolvimento do Portal do Software Público Brasileiro, em que focamos para este trabalho no estudo das práticas de desenvolvimento dos cenários de testes de aceitação, prototipação, avaliação de heurísticas de usabilidade, aplicação de \textit{checklists} de usabilidade; assim como a relação entre essas práticas.

A partir do estudo de caso, observamos que existe uma certa dificuldade em aderir tanto práticas de testes, quanto práticas de usabilidade pela equipe de desenvolvimento, principalmente durante sprints de final de release, onde o prazo de entrega encontra-se mais perto, e a equipe tende a se preocupar mais com o desenvolvimento da funcionalidade em si. Porém quando essas práticas são utilizadas de forma frequente e como planejadas, verificamos que o desenvolvimento dos testes de aceitação , assim como o uso de protótipos, pode agilizar o processo de teste de usabilidade por meio de avaliações e \textit{checklists}, pois não existe a necessidade de uma funcionalidade está totalmente desenvolvida. Entretanto é necessário dar continuidade aos testes de usabilidade durante todo o procesos de desenvolvimento, para que a usabildade possa evoluir juntamente com o processo de desenvolvimento.



