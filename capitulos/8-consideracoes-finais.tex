\chapter{Conclusão}
\label{consideracoes-finais}

Durante esta primeira parte do trabalho de conclusão de curso, estudamos o desenvolvimento de software empírico e suas características, assim como a forma que esse desenvolvimento está ligado com testes automatizados e como as práticas de testes podem ser aplicadas ou não em um ambiente real de desenvolvimento que já se encontra estabelecido, além da aplicação de práticas de melhorias de usabilidade através dos cenários de testes automáticos.

A partir do estudo de caso, observamos que existe uma certa dificuldade em aderir tanto práticas de testes, quanto práticas de usabilidade pela equipe de desenvolvimento, principalmente durante sprints de final de release, onde o prazo de entrega encontra-se mais perto, e a equipe tende a se preocupar mais com o desenvolvimento da funcionalidade em si.
