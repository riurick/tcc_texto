%TODO rever referencias
\chapter{2. Testes}
\label{cap:testes}

Testar é uma pratica intrínseca ao desenvolvimento e é antiga a necessidade de
cirar programas para testar cenários específicos~\cite{everett2007}.
%
A automação de testes é uma prática ágil, eficaz e de baixo custo para melhorar
a qualidade dos sistemas de software.

No entanto utilizar testes automatizados 
como uma premissa básica do desenvolvimento é um fenômeno relativamente recente, ]
com início em meados  da década de 1990~\cite{cotter1995}.
%
Além do fato de ser uma técnica bastante utilizada pelas metodologias ágeis
de desenvolvimento.

%------------------------------------------------------------------------------%

\section{Testes Automatizados}

Testes automatizados é a prática de tornar os testes de software independentes da
intervenção humana, criando scripts ou programas simples de computador que exercitam 
o sistema em teste, capturam os efeitos colaterais e fazem verificações, tudo 
automaticamente e dinamicamente~\cite{meszaros2007}.

Os testes automatizados afetam diretamente a qualidade dos sistemas de software,
portanto agregam valor  ao produto final, mesmo que os artefatos adicionais
produzidos não sejam visíveis para os usuários finais do sistemas.
%
Estes testes podem ser divididos em diversos tipo, o que facilita a manutenção 
dos mesmos, coleta de métricas.

\begin{enumerate}

\item \textbf{Testes de unidade:} teste de correção responsável por testar os 
menores trechos de código de um sistema que possui um comportamento definido e nomeado.
%
Normalmente, ele é associado a funções para linguagens procedimentais e métodos em 
linguagens orientadas a objetos.
\item \textbf{Testes funcionais:}
%ESCREVER
\item \textbf{Testes de integração:}denominação ampla que representa a busca de erros 
de relacionamento entre quaisquer módulo de um software, incluindo desde a integração 
de pequenas unidades até a integração de bibliotecas das quais um sistema depende, 
servidores e gerenciadores de banco de dados.
\item \textbf{Testes de interface de usuário} testes que verificam a correção por meio 
da simulação de eventos de usuário, a partir destes eventos, são feitas verificações 
na interface e em outras camadas.
\item \textbf{Testes de leiaute:} testes que buscam avaliar a beleza da interface e 
verificar a presença de erros após a renderização, dificeis de indentificar com testes 
comuns de interface
\item \textbf{Testes de aceitação:} são testes de correção e validação, idealmente 
especificados por clientes ou usuários finais do sistema para verificar se um modulo 
funciona como foi especificado~\cite{martin2005}.
%
Testes de aceitação devem utilizar linguagem proxima da natural para evitar problemas 
de interpretação e de ambiguidades~\cite{cunningham2005}.
\item \textbf{Testes de desempenho:} testes que executam trechos do sistema e armazenam 
os tempos de duração obtidos, que ajudam a identificar gargalos que precisam de otimização 
para diminuir o tempo de resposta  para o usuario~\cite{liu2009}.
\item \textbf{Testes de carga:}  teste que exercita o sistema sobre condições de uso 
intenso para avaliar se a infraestrutura é adequada para a expectativa de uso do sistema.~\cite{avritze1994}
\item \textbf{Testes de estresse:} teste que visa descobrir os limites do uso da infraestrutura, 
isto é , qual a quantidade máxima de usuários e requisições que o sistema consegue antender 
corretamente e em um tempo aceitável.
\item \textbf{Testes de longevidade:} teste que tem por objetivo encontrar erros somente 
visiveis com um\item longo tempo de execuçao do sistema, erros que podem ser de cache, replicação, 
execução de serviços agendados, vazamento de memória.
\item \textbf{Testes de segurança:} os testes de segurança servem para verificar se os 
dados ou funcionalidades confidenciais de um sistema  estão protegidos de fraude ou de 
usuários não autorizados. A segurança de um sofware pode envolver aspectos de confidenciabilidade, 
integridade, autenticação, autorização, privacidade~\cite{whittaker2006}.




\end{enumerate}

%------------------------------------------------------------------------------%
\section{Técnicas de Desenvolvimento de testes automatizados}

Automação de testes é uma técnica voltada principalmente para a melhoria de qualidade dos sistemas de software. 
%
No processo de desenvolvimento de software é fundamental controlar o custo do processo de testes, para isso baterias de testes automatizados devem ser bem definidas e implementadas. Assim é importante conhecer boas práticas e técnicas de de desenvolvimento de testes automatizados.    
%
Existem várias técnicas de desenvolvimento de software com testes que influenciam diretamente na qualidade do sistema. Estas técnicas geralmente possuem um processo de atividades pequeno e simples, como TDD e BDD.

\subsection{TDD - Test Driven Development}

Desenvolvimento dirigido por testes, também conhecido como TDD (Test-Driven Develepment) é uma técnica de desenvolvimento de software que se dá pela repetição dosciplinada de um ciclo curto de passos de implementação de testes e do sistema~\cite{koskela2007}.
%
O ciclo de TDD é definido pelos seguintes passos:
%
\begin{enumerate}
\item Implementar um caso de teste;
\item Implementar um trecho do código suficiente para o novo caso de teste ter sucesso de tal modo que não quebre os testes previamente escritos;
\item Se necessário, refatorar o código produzido para que ele fique mais organizado;
\end{enumerate}
%

\subsubsection {Benefícios do TDD}

Uma boa prática do TDD é a bateria de testes, que ajuda o desenvolvedor a evitar erros de regreção, quando o desenvolvimento de uma nova funcionalidade quebra uma já existente. TDD também tende a contribuir com uma alta cobertura de código, uma fez que o desenvolvedor precisa escrever o testes antes da funcionalidade, possibilitando a criação de um código mais preciso, coeso e menos acoplado. 
%
Para Massol em JUnit in Action, “o objetivo de TDD é ‘código claro que funciona’~\cite{massol2003}.
%
TDD propõe o desenvolvimento sempre em pequenos passo, deve-se escrever testes sempre para uma menor funcionalidade possível, escrever o código mais simples que faça o teste passar e fazer sempre apenas uma refatoração por vez~\cite{beck2002}. Assim o desenvolvedor se detém a criar soluções simples, sempre acompanhado de um constante feedback dos testes.
%
O ciclo curto de passos definidos por TDD cria uma dependência forte entre codificação e testes, o que favorece e facilita a criação de sistemas com alta testabilidade~\cite{bernardo2011}. Índices altos de cobertura de código e testabilidade não garantem necessariamente qualidade do sistema, mas são métricas bem vistas para sistemas bem desenvolvidos.

\subsection{BDD - Behavior Driven Development}


