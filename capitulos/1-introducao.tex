\chapter{Introdução}

O desenvolvimento de software usando métodos empíricos, como software livre e métodos ágeis, é uma realidade. Uma prática básica de métodos ágeis são os testes automatizados, preocupando-se com as aplicações de testes em sistemas cada vez mais dinâmicos, com propostas para aumento de qualidade e produtividade de software~\cite{vicente2010}.
%
Adicionalmente, as áreas relacionadas a usabilidade vem sendo estudadas para serem aplicadas no contexto de desenvolvimento empírico de software. Criar software que seja útil e fácil de usar é um fator importante para a evolução dos sistemas ~\cite{santos2012}.

Um dos grandes problemas encontrados em projetos de software livre é a pouca atenção dada aos aspectos referentes a usabilidade das aplicações. Para ~\citeonline{moreira} enfatizar na criação, melhoria e teste do código fonte pode ser um dos principais problemas que contribui para a falta de usabilidade dos softwares livres.

%
Segundo ~\citeonline{preece2007}, uma das causas da baixa adoção de softwares livres em mercados de larga escala é a baixa qualidade dos estilos de interação implementados nas interfaces dos produtos. Uma maioria não se preocupa com bons elementos de interface com usuário (UI). 

Atualmente, algumas comunidades de software livre vêm se atentando as questões de usabilidade como forma de se manter no mercado. 
%
O Noosfero, um sistema para desenvolvimento de redes sociais na qual faz parte este estudo, tem tido algumas iniciativas por parte dos desenvolvedores da plataforma para a melhoria da usabilidade da ferramenta. 

A adoção das práticas de usabilidade no contexto de métodos ágeis vêm sendo estudada por vários autores no qual afirmam que é possível a integração visto que tanto os métodos ágeis como os processo de usabilidade têm em comum características que colocam o foco do desenvolvimento nas necessidades e anseios dos usuários finais, na interação entre os \textit{stakeholders} envolvidos e na qualidade final do produto a ser desenvolvido.

A automação de testes é uma prática ágil, eficaz e de baixo custo que também busca melhorar a qualidade dos sistemas de software~\cite{cotter1995}. O que este trabalho busca é encontrar um ponto de intersecção entre os testes automatizados e a usabildiade, utilizados no desenvolvimento ágil.

 

%------------------------------------------------------------------------------%
\section{Problemas}

Um dos problemas encontrados nos softwares livres é a baixa usabilidade de suas interfaces, o que resulta na perda de usuários. 
%
Os desenvolvedores de software livre possuem uma mentalidade mais voltada para a funcionalidade do que para os usuários do sistema~\cite{santos2012}. 

Outro problema do desenvolvimento de software atualmente está em garantir qualidade de software de forma prática e econômica, testes podem interferir nestas características se não forem bem planejados.

\section{Objetivos}

\subsection{Objetivo geral}
	 
Este trabalho de conclusão de curso busca analisar técnicas e métodos de usabilidade, assim como técnicas de desenvolvimento e de testes de software no desenvolvimento empírico de software para verificarmos os possíveis efeitos de técnicas como TDD (desenvolvimento dirigido por testes) e BDD (desenvolvimento dirigido por comportamento) na usabilidade de um software, a partir de técnicas como testes de usabilidade, dentre outras.

	 
\subsection{Objetivos específicos}

Para satisfazer os objetivos gerais do trabalho foram definidos objetivos específicos abaixo:

\begin{enumerate}
\item Verificar aplicação das práticas do BDD e TDD no processo de desenvolvimento de software livre;

\item Integrar usabilidade no ciclo de vida de desenvolvimento de software livre;
\item Identificar quais técnicas de \emph{design} e usabilidade  podem ser utilizadas em cada fase do processo ágil de desenvolvimento de software livre;
\item Verificar como práticas de BDD e TDD podem influenciar a usabilidade de um software.


\end{enumerate}

\section{Questões de pesquisa}

Esta seção descreve as questões de pesquisa levantadas durante o trabalho.
	
\begin{enumerate}
\item Como os testes automatizados são definidos e implementados em um ambiente de desenvolvimento de software empírico?
\item Como inserir os princípios de usabilidade dentro do processo ágil de desenvolvimento de software?
\item Como o processo de desenvolvimento utilizando práticas do BDD e TDD podem influenciar os testes de usabilidade?
%\item Projetos de software centrados no usuário que possuem avaliação de usabilidade desde o seu início apresentam melhores resultados no teste aceitação?
\end{enumerate}
%------------------------------------------------------------------------------%
%-------------------  CONTINUAR REVISAO TCC2 ------------------------%
\section{Metodologia}

%TODO: DEFINIR METODOLOGIA DE ESTUDO DE CASO. ESSE TOPICO EH NECESSARIO??? LEVANDO EM CONTA QUE A METODOLOGIA TAMBEM ESTÁ NO CAPITULO DE ESTUDO DE CASO.

\subsection{Classificação da Pesquisa}

Neste trabalho, a coleta e análise dos dados serão realizadas com base em  materiais já publicados, constituído principalmente de livros, artigos de periódicos e  materiais disponibilizados na Internet, caracterizando-se, portanto, como uma pesquisa bibliográfica do ponto de vista do procedimento técnico empregado~\cite{gil1991}.


\subsection{As etapas da Pesquisa}

\begin{description}
\item[Estudo bibliográfico]

Levando em consideração os objetivos de pesquisa, o estudo bibliográfico  aborda a usabilidade e sua relação com o desenvolvimento de testes, tentando entender como a usabilidade está presente em nosso meio. As áreas pesquisadas são Arquitetura de Informação, Automação de Testes, Design, Ergonomia, Desenvolvimento Empírico, Interação Humano Computador, Engenharia de Usabilidade, Experiência do Usuário, entre outras.

%
Estamos estudando os métodos ágeis de desenvolvimento e o software livre como métodos de desenvolvimento de software, aplicando este trabalho em um cenário real de desenvolvimento da plataforma de redes colaborativas Noosfero.

Além disso é necessário conhecer o ciclo de \emph{design} centrado no usuário. Encontrar maneiras de como inserir os métodos de usabilidade utilizando uma abordagem ágil em um contexto de software livre, e verificar como as técnicas já utilizadas atacam as características de usabilidade de um software.
%

\item[Coletas de dados:]
%

Para avaliação de usabilidade, os dados serão coletados através de questionários e entrevistas que serão feitas através de experimentos realizados em um estudo de caso no portal da participação social.
Para verificação do desenvolvimento de testes métricas de testes serão coletadas durante o processo de desenvolvimento em que este trabalho está baseado.

\end{description}
\end{comment}
\section{Organização do trabalho}

Nesta seção está apresentada a organização deste trabalho, e o que se refere cada capítulo.

O estudo inicia-se no segundo capítulo, em que se aborda o desenvolvimento empírico, onde dissertamos sobre software livre assim como uma breve base conceitual sobre metodologias ágeis de desenvolvimento.

O terceiro capítulo aborda conceitualmente as técnicas de testes automatizados que geralmente são utilizadas no processo empírico, além dos tipos de testes que podem ser aplicados. Nesta seção são destacados o TDD (Test Driven Development) e o BDD (Behavior Driven Development).

O quarto capítulo aborda os principais conceitos e as áreas referentes à usabilidade, além de a abordar os processos da engenharia de usabilidade e interação humano computador e as formas de como poder inserir as técnicas no ciclo de vida de desenvolvimento empírico.

O quinto capítulo aborda o estudo de caso realizado no Portal do Software Público, baseado na plataforma Noosfero.

O trabalho se encerra com as considerações finais, assim como as análises obtidas de todo o processo do trabalho.



