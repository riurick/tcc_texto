	\chapter{Introdução}

	O impacto e utilização do software livre na indústria de software tem aumentando gradativa-
	mente nas últimas décadas e influenciado significativamente a economia global~\cite{benkler2006}. 
	%
	O desenvolvimento de software livre baseado em métodos ágeis é uma realidade. Metodologias ágeis 
	como Scrum e XP recomendam que todas as pessoas envolvidas em um projeto trabalhem controlando a 
	qualidade do produto todos os dias, pois baseiam-se na ideia de que prevenir defeitos é mais barato
	que identificá-los, assim é recomendado testes automatizados para ajudar a garantir a qualidade dos 
	sistemas desenvolvidos~\cite{bernardo2011}.
	%
	A área de teste de software também tem evoluido, preocupando-se com as aplicações de testes
	em sistemas cada vez mais dinâmicos e reutilizáveis com propostas para aumento de qualidade
	e produtividade de software~\cite{vicente2010}.
	%------------------------------------------------------------------------------%
	\section{Problemas}
	Um dos grandes problemas encontrados nos softwares livres é a baixa usabilidade de suas interfaces, o que resulta na perda de usuários. 
	%
	Os desenvolvedores de software livre possuem uma mentalidade mais voltada para a funcionalidade do que para os usuários do sistema [Thor08]. Possuem código de qualidade, com algarismos eficientes e de bom desempenho e são produzidos por desenvolvedores motivados e voluntários. [Ana Paula] 
	%
	Outro problema do desenvolvimento de software atualmente está em garantir qualidade de software de 
	forma prática e economica.

	\section{Objetivos}

	\subsection{Objetivos Gerais}
	 
	Este trabalho de conclusão de curso busca analisar técnicas e metodos des usabilidade, assim como técnicas de desenvolvimento e de testes de software em um processo empírico para verificar a os possiveis efeitos de técnicas como TDD e BDD na usabilidade de um software, a partir de técnicas como testes de usabilidade, dentre outras.
	
	 
	%------------------------------------------------------------------------------%
	 
	\subsection{Específicos}

	 Para satisfazer os objetivos gerais do trabalho foram definidos objetivos específicos:

	\begin{enumerate}
	\item Verificar aplicação das práticas do TDD no processo de desenvolvimento de novos recursos na plataforma Noosfero;
	\item Verificar aplicação das práticas do BDD no processo de desenvolvimento de novos recursos na plataforma Noosfero;
	\item Integração da Engenharia de Usabilidade no ciclo de vida de desenvolvimento de softwrare livre;
	\item Identificar quais as técnicas de design e usabilidade  podem ser utilizadas em cada fase do processo ágil de desenvolvimento de software livre;
	\item Analisar a usabilidade dos portais governamentais utilizando diferentes técnicas de avaliação, a fim de que a avaliação possa ser feita não somente pelo especialista de usabilidade, mas também para que os usuários avaliem a qualidade em uso dos portais;
	\item Evouluir o processo de desenvolvimento da plataforma Noosfero;
	\end{enumerate}

	\section{Questões}
	\begin{enumerate}
	\item Sistemas que são desenvolvidos pensando no usuário desde o inicio do processo apresenta melhor  satisfação de uso?
	\item Como são definidos e implementados testes automatizados em um ambiente de desenvolvimento de software empírico?
	\item É possível inserir os princípios de usabilidade dentro do processo ágil de desenvolvimento de software?
	\item As funcionalidades desenvolvidas na plataforma Noosfero utilizando práticas do BDD e TDD tem melhores resultados em testes de usabilidade?
	\item projetos que foram centrados no usuário e que passaram por avaliação de usabilidade desde o início do projeto apresentam um melhor resultado no teste aceitação?
	\end{enumerate}
	%------------------------------------------------------------------------------%
	 
	\section{Organização do Trabalho}
	
	
	O segundo capítulo aborda o desenvolvimento de software empírico, onde dissertamos sobre software livre 
	e suas licenças assim como uma breve base conceitual sobre metodologias ágeis de desenvolvimento.
	% 
	O terceiro capítulo aborda conceitualmente as técnicas de testes automatizados que geralmente
	são utilizadas no processo empírico, além dos tipos de testes que podem ser aplicados. Nesta seção são 
	destacados o TDD (Test Driven Development) e o BDD (Behavior Driven Development).
	%
	O sétimo cápítulo aborda a plataforma do Noosfero, e as funcionalidades desenvolvidas para a Rede Comunidade UnB\footnote{\url{comunidade.unb.br}}, a partir desta plataforma.
	%
	O trabalho se encerra com os resultados e análises obtidas de todo o processo e discussão dos próximos passos


