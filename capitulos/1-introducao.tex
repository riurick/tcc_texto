\chapter{Introdução}

O desenvolvimento de software usando métodos empíricos, como software livre e métodos ágeis, é uma realidade. Uma prática básica de métodos ágeis são os testes automatizados, preocupando-se com as aplicações de testes em sistemas cada vez mais dinâmicos, com propostas para aumento de qualidade e produtividade de software~\cite{vicente2010}.
%
Adicionalmente, as áreas relacionadas a usabilidade vem sendo estudadas para serem aplicadas no contexto de desenvolvimento empírico de software. Criar software que seja útil e fácil de usar é um fator importante para a evolução dos sistemas ~\cite{santos2012}.

Por exemplo, um problema encontrado em projetos de software livre é a pouca atenção dada aos aspectos referentes a usabilidade das aplicações. Para ~\citeonline{moreira} enfatizar na criação, melhoria e teste do código fonte pode ser um dos principais problemas que contribui para a falta de usabilidade dos softwares livres.
%
Ainda, segundo ~\citeonline{preece2007}, uma das causas da baixa adoção de softwares livres em mercados de larga escala é a baixa qualidade dos estilos de interação implementados nas interfaces dos produtos. 

Atualmente, algumas comunidades de software livre vêm se atentando à usabilidade como forma de manter o software no mercado. 
%
O Noosfero\footnote{\url{noosfero.org}}, um sistema para desenvolvimento de redes sociais na qual faz parte este estudo, tem tido algumas iniciativas por parte dos desenvolvedores da plataforma para a melhoria da usabilidade da ferramenta. 

A adoção das práticas de usabilidade no contexto de métodos ágeis vêm sendo estudada por vários autores no qual afirmam que é possível a integração visto que tanto os métodos ágeis como os processo de usabilidade têm em comum características que colocam o foco do desenvolvimento nas necessidades e anseios dos usuários finais, na interação entre os \textit{stakeholders} envolvidos e na qualidade final do produto a ser desenvolvido.

A automação de testes é uma prática ágil, eficaz e de baixo custo que também busca melhorar a qualidade dos sistemas de software~\cite{cotter1995}. O que este trabalho busca é encontrar um ponto de intersecção entre os testes automatizados e a usabilidade, utilizados no desenvolvimento ágil.

 

%------------------------------------------------------------------------------%
\section{Problemas}

Um dos problemas encontrados no desenvolvimento empírico é a baixa usabilidade de suas interfaces, o que resulta na perda de usuários. 
%
Os desenvolvedores possuem uma mentalidade mais voltada para a funcionalidade do que para os usuários do sistema~\cite{santos2012}. 
%
Outro problema do desenvolvimento empírico está em garantir qualidade de software de forma prática e econômica, testes podem interferir nestas características se não forem bem planejados.

A partir destes problemas levantamos as seguintes questões:
	
\begin{enumerate}
\item Como os testes automatizados são definidos e implementados em um ambiente de desenvolvimento de software empírico?
\item Como inserir os princípios de usabilidade no desenvolvimento empírico de software?
\item Como as práticas do BDD podem se relacionar com as práticas de usabilidade?
\end{enumerate}

\section{Objetivos}
	 
Durante este trabalho de conclusão de curso buscamos analisar técnicas e métodos de usabilidade, assim como técnicas de desenvolvimento de testes no desenvolvimento empírico de software para verificarmos as possíveis relações de técnicas de BDD (desenvolvimento dirigido por comportamento) na usabilidade de um software.

Para satisfazer os objetivos gerais do trabalho definimos objetivos específicos abaixo:

\begin{enumerate}
\item Verificar aplicação das práticas do BDD no processo de desenvolvimento empírico de software;
\item Integrar práticas de usabilidade no ciclo de vida de desenvolvimento empírico de software;
\item Identificar quais técnicas de usabilidade podem ser utilizadas em cada fase do desenvolvimento empírico.
\item Verificar a relação de práticas de BDD e práticas de usabilidade no desenvolvimento empírico de software.
\end{enumerate}

\section{Organização do Trabalho}


Nesta seção apresentamos a organização deste trabalho, e o que se refere cada capítulo.
%
O estudo inicia-se no segundo capítulo, em que se aborda o desenvolvimento empírico, onde dissertamos sobre software livre assim como uma breve base conceitual sobre metodologias ágeis de desenvolvimento.
%
O terceiro capítulo aborda conceitualmente as técnicas de testes automatizados que geralmente são utilizadas no processo empírico, além dos tipos de testes que podem ser aplicados. Nesta seção destaca-se as práticas do BDD (Behavior Driven Development).
%
O quarto capítulo aborda os principais conceitos e as áreas referentes à usabilidade, além de a abordar os processos da engenharia de usabilidade e interação humano computador e as formas de como poder inserir as técnicas no ciclo de vida de desenvolvimento empírico.
%
O quinto capítulo aborda o estudo de caso realizado no Portal do Software Público, baseado na plataforma Noosfero, assim como os resultados obtidos.
%
O trabalho se encerra com as conclusões, assim como as análises obtidas de todo o processo do trabalho.

