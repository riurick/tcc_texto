	\chapter{Introdução}

	O impacto e utilização do software livre na indústria de software tem aumentando gradativamente nas últimas décadas e influenciado significativamente a economia global ~\cite{benkler2006}. 
	%
	O desenvolvimento de software livre baseado em métodos ágeis é uma realidade. Metodologias ágeis como Scrum e XP recomendam que todas as pessoas envolvidas em um projeto trabalhem controlando a qualidade do produto todos os dias, pois baseiam-se na ideia de que prevenir defeitos é mais barato que identificá-los, assim é recomendado testes automatizados para ajudar a garantir a qualidade dos sistemas desenvolvidos ~\cite{bernardo2011}.
	%
	A área de teste de software também tem evoluido, preocupando-se com as aplicações de testes	em sistemas cada vez mais dinâmicos e reutilizáveis com propostas para aumento de qualidade e produtividade de software~\cite{vicente2010}.
	%------------------------------------------------------------------------------%
	\section{Problemas}

	Um dos grandes problemas encontrados nos softwares livres é a baixa usabilidade de suas interfaces, o que resulta na perda de usuários. 
	%
	Os desenvolvedores de software livre possuem uma mentalidade mais voltada para a funcionalidade do que para os usuários do sistema [Thor08]. Possuem código de qualidade, com algarismos eficientes e de bom desempenho e são produzidos por desenvolvedores motivados e voluntários. [Ana Paula] 
	%
	Outro problema do desenvolvimento de software atualmente está em garantir qualidade de software de 
	forma prática e economica.

	\section{Objetivos}

	Nesta seção iremos abordar sobre os objetivos deste trabalho de conclusão de curso, dissertando tanto a respeito dos objetivos gerais do trabalho quanto dos objetivos específicos.

	\subsection{Objetivos Gerais}
	 
	Este trabalho de conclusão de curso busca analisar técnicas e metodos des usabilidade, assim como técnicas de desenvolvimento e de testes de software em um processo empírico para verificar a os possiveis efeitos de técnicas como TDD e BDD na usabilidade de um software, a partir de técnicas como testes de usabilidade, dentre outras.
	
	 
	%------------------------------------------------------------------------------%
	 
	\subsection{Objetivos Específicos}

	 Para satisfazer os objetivos gerais do trabalho foram definidos objetivos específicos:

	\begin{enumerate}
	\item Verificar aplicação das práticas do TDD no processo de desenvolvimento de novos recursos na plataforma Noosfero;
	\item Verificar aplicação das práticas do BDD no processo de desenvolvimento de novos recursos na plataforma Noosfero;
	\item Integração da Engenharia de Usabilidade no ciclo de vida de desenvolvimento de softwrare livre;
	\item Identificar quais as técnicas de design e usabilidade  podem ser utilizadas em cada fase do processo ágil de desenvolvimento de software livre;
	\item Analisar a usabilidade dos portais governamentais utilizando diferentes técnicas de avaliação, a fim de que a avaliação possa ser feita não somente pelo especialista de usabilidade, mas também para que os usuários avaliem a qualidade em uso dos portais;
	\item Evouluir o processo de desenvolvimento de software empírico a partir de um estudo de caso específico;
	\end{enumerate}

	\section{Questões}
	\begin{enumerate}
	\item Sistemas que são desenvolvidos pensando no usuário desde o início do processo apresenta melhor  satisfação de uso?
	\item Como os testes automatizados são definidos e implementados em um ambiente de desenvolvimento de software empírico?
	\item É possível inserir os princípios de usabilidade dentro do processo ágil de desenvolvimento de software?
	\item O processo de desenvolvimento utilizando práticas do BDD e TDD apresentam melhores resultados em testes de usabilidade?
	\item Projetos de software centrados no usuário que possuem avaliação de usabilidade desde o seu início apresentam melhores resultados no teste aceitação?
	\end{enumerate}
	%------------------------------------------------------------------------------%
	 


\section{Metodologia}

\subsection{Classificação da Pesquisa}

Neste trabalho, a coleta e análise dos dados serão realizadas com base em  materiais já publicados, constituído principalmente de livros, artigos de periódicos e  materiais disponibilizados na Internet (GIL, 1991); caracterizando-se, portanto, como uma pesquisa bibliográfica do ponto de vista do procedimento técnico empregado. 

Do ponto de vista da natureza, a pesquisa é aplicada pois tem como objetivo gerar conhecimentos para aplicação prática, drigida à solução de problemas específicos. Do ponto de vista da forma de abordagem do problema será tanto qualitativa como quantitativa.

\subsection{As etapas da Pesquisa}

	\begin{enumerate}
	\item \textbf{Estudo bibliográfico}

	Levando em consideração os objetivos da pesquisa, o estudo bibliográfico aborda a usabilidade e a sua relação com as diferentes áreas de conhecimento, tentando entender como a usabilidade está presente em nosso meio. As áreas pesquisadas são (Arquitetura da Informação, Design, Ergonomia, Interação Humano Computador, Engenharia de Usabilidade, Experiência do Usuário, Psicologia, entre outras).

Deve ser feito um estudo sobre o processo  de desenvolvimento de software livre. Além disso é necessário conhecer o ciclo de design centrado no usuário que é a base de toda a pesquisa. Encontrar maneiras de como inserir os métodos de usabilidade dentro do contexto de software livre. 

	Pesquisa dos principais paradigmas de avaliação de usabilidade, descrevendo as técnicas utilizadas e sua aplicação para a engenharia de software.

	\item \textbf Coletas de dados

	Os dados serão coletados através de questionários e entrevistas que serão feitas através de experimentos realizados em um estudo de caso no portal da participação social.

	\end{enumerate}

	\section{Organização do Trabalho}

	Nesta seção está apresentada a organização deste trabalho, e o que se refere cada capítulo.
	
	O segundo capítulo aborda o desenvolvimento de software empírico, onde dissertamos sobre software livre e suas licenças assim como uma breve base conceitual sobre metodologias ágeis de desenvolvimento.
	
	O terceiro capítulo aborda conceitualmente as técnicas de testes automatizados que geralmente são utilizadas no processo empírico, além dos tipos de testes que podem ser aplicados. Nesta seção são destacados o TDD (Test Driven Development) e o BDD (Behavior Driven Development).

	O quarto capítulo aborda os principais conceitos e as áreas referentes a usabilidade e a as técnicas existentes utilizadas na engenharia de usabilidade e áreas afins.

	O quinto capítulo aborda os ciclos de vida da engenharia de usabilidade e interação humano computador e formas de como poder inserir as técnicas no ciclo de vida de desenvolvimento de software empírico.

	O sexto capítulo é um estudo de caso do Portal da Participação Social, onde foi planejado a utilização de algumas técnicas para avaliação do portal.

	O sétimo cápítulo aborda a plataforma do Noosfero, e as funcionalidades desenvolvidas para a Rede Comunidade UnB\footnote{\url{comunidade.unb.br}} e o Portal da FGA\footnote{\url{www.fga.unb.br}}, a partir desta plataforma.

	O trabalho se encerra com os resultados parciais, assim como as análises obtidas de todo o processo do trabalho, aĺém de uma discussão para os próximos passos.

	


