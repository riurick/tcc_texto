\chapter{Estudo Experimental}

\section{Estudo de Caso - Portal da Participação Social}

\subsection{Identificação}

\textbf{Título}:“Avaliação da usabilidade do portal participa.br”.

\textbf{Tema}: “Avaliação da Usabilidade” 

\textbf{Área técnica}: “Qualidade de Software” 

\textbf{Autor:} Jônatas Medeiros de Mendonça  

\textbf{Afiliação:} FGA/UnB 

\textbf{Local:} Brasília – Brasil - Data:  21/03/2014 

\subsection{Caracterização}

\textbf{Nome da empresa:} Presidência da República

\textbf{Domínio:} Análise do usuário participa.br

\textbf{Tecnologias:} Noosfero, Rails 

\textbf{Plataforma:} Linux 

\textbf{Equipe: A equipe do projeto é constituída por 1 professor orientador e 1 aluno.} 

\textbf{Alocação da equipe ao projeto:} 

\textbf{Orientador:} Paulo Meirelles

\textbf{Aluno pesquisador}: Jônatas Medeiros de Mendonça


\subsection{Introdução}

O Portal da participação Social é um portal que agrega informações sobre oportunidades de participação social no governo federal e estimula a formação de comunidades em torno de temas ligados à participação. Informa sobre as consultas públicas, oferece ambientes para interação em vídeo e chat em eventos de governo. É um repositório das metodologias das conferências de políticas públicas. O Portal capta demandas da sociedade que não passem, necessariamente, pelos fluxos formais de participação. É uma plataforma para ampliar o debate entre a sociedade civil e o governo. [Citar fonte	

\subsection{Justificativa}

	No ano 2000 foi criado o programa de Governo Eletrônico (e-GOV) que têm como principais objetivos democratizar o acesso a informação e dinamizar a prestação de serviços públicos eletrônicos com foco em eficiência e efetividade das funções governamentais prestadas ao cidadão. ()

	O Governo Federal adotou várias práticas de padrões para melhorar o acesso e a divulgação das informações e dos serviços do 
governo eletrônico, respeitando as particularidades de cada cidadão. Por isso foi criado o Modelo de Acessibilidade do Governo Federal (e-MAG) que 
contém um conjunto de recomendações para tornar os sítios e portais acessíveis para uma maior quantidade de pessoas possíveis. 

	O objetivo era de estabelecer padrões de qualiedade de uso, desenho, arquitetura da informação e navegação, desenvolvimento e manutenção na gestão dos sítios governamentais.O  e-PWG ~\footnote{e-PWG <portal>}, como ficou conhecido, são padrões web para o Governo eletrônico.

	Foi feita uma análise do sítio participa.br em parceria com o Instituto Federal do Rio Grande do Sul (IFRS) onde foi criado um relatório para orientar e sugerir correções que facilitarão o acesso ao seu conteúdo quanto as recomendações sobre o Modelo de Acessibilidade em Governo Eletrônico (e-MAG) e com os Padrões Web em governo Eletrônico (e-PWG). Essas recomendações busca levantar problemas que impactarão na experiência de uso do portal pelo cidadão.


	Como já existia um estudo referente sobre a Acessibilidade do portal participa.br o nosso estudo se deu em avaliar a qualidade em uso do portal no sentido de verificar a satifação dos usuários ao utilizar o portal.

	Primeiramente foi preciso realizar um checklist de usabilidade para verificar os possíveis problemas no portal antes de realizar os testes com os usuários.Foi utilizado as heurísticas de Nielsen para identificação dos problemas de usabilidade.


\section{Definição do Estudo Experimental}

\subsection{Objetivo Global}

	Analisar a interação dos usuários com o portal participa.br a fim de avaliar a qualidade em uso dos usuários com este portal. 

\subsection{Objetivos de Medição}

\begin{itemize}
\item Conhecer quem são os usuários do Portal da Participação Social.
\item Avaliar de forma subjetiva o grau de satisfação dos usuários com a utilização do portal participa.br. 
\item O objeto de estudo deste experimento é a aplicação do paradigma Teste de usabilidade e da técnica de avaliação da satisfação do usuário através de questionários.
\end{itemize}

\subsection{Objetivo do Estudo}


\begin{table}[h]
\begin{tabular}{|l|l|}
\hline
Analisar             & Portal da Participação Social (participa.br) \\ \hline
Com propósito de     & Avaliar Qualidade em Uso (ISO/IEC 9126-4)    \\ \hline
Com respeito ao      & Satisfação do usuário                        \\ \hline
Do ponto de vista de & Usuário                                      \\ \hline
No contexto de       & Portais Governamentais                       \\ \hline
\end{tabular}
\end{table}

\subsection{Questões}

A partir do objetivo de medição estabelecido no quadro 1  foram definidas questões sobre o que é preciso saber de forma a apoiá-la a entender se o objetivo específico foi alcançado, e para cada questão foram definidas as métricas relacionadas no quadro 2: 

\begin{table}[h]
\begin{tabular}{|l|l|l|}
\hline
\textbf{Questões}                                                          & Métricas                      & Diretrizes para interpretação                                                       \\ \hline
Q1. Qual o perfil do usuário que utiliza o portal participa.br?            &                               & Análise de Dados Estatísticos, criação de personas, análise dos dados qualitativos. \\ \hline
Q2. Qual o grau de satisfação do usuário que utiliza o portal participa.br & Grau de satisfação do usuário & Escore da satisfação global pelo usuário (OVERALL)                                  \\ \hline
Q3. O portal participa.br garante a participação social da população?      &                               & De acordo com as respostas dadas aos cenários do experimento.                       \\ \hline
\end{tabular}
\end{table}


\subsection{Questões que não podem ser respondidas pelo estudo experimental}

\section{Metodologia}

\subsection{Análise do Perfil dos Usuários}

	Foi levantada algumas técnicas na qual podemos identificar o perfil dos usuários do portal da Participação Social.

\subsubsection{Dados Estatísticos (Google analytcs, outros) - Pesquisa quantitativa}

	Através dos dados estatisticos é possivel identificar algumas informações sobre o perfil dos usuários que acessam o portal. Nas pesquisas quantitativas não são necessários o contato direto com o usuário. Esses dados estatísticos podem ser coletados de base de dados, redes sociais ou sistemas de análises de sites.

\subsubsection{Questionário de identificação de perfil dos usuários.}

Para identificar o perfil dos usuários do Portal da Participação social é necessário realizar uma pesquisa qualitativa para levantamento das principais caracterísiticas contextuais dos usuários típicos, de modo a comprender quem são, qual o conhecimento e experiência com a internet e como utilizam para realizar seu trabalho acadêmico ou profissional. 

O portal deve atingir a população brasileira, no entanto sabemos que a grande maioria das pessoas que acessam ao portal são pessoas engajadas em algum projeto social, manifestações ou mobilizações de cunho político-sociais. A realização dessa pesquisa será feita com os usuários do Portal da Participação Social \footnote{\url{participa.br}}

A análise do questionário servirá para entender o perfil de usuários do Portal da participação social, através da investigação de seus interesses.


\subsubsection{Identificação de Personas}

	Para a definição de usuários podemos utilizar a técnica de “Persona” que são personagens fícticios criados com base em dados reais. Os Personas atuam como representantes dos usuários reais e representam as necessidades de um grupo maior. 

	A utilização de Personas permite ter um maior foco no usuário, deixando o projeto centrado no usuário. É utilizado para a identificação de requisitos, criação de cenários e user stories. 

	Para podermos identificar os personas primeiramente temos que realizar uma pesquisa quantitativa no qual podemos identificar os grupos de usuários. Após a identificação dos grupos de usuários é realizado a pesquisa qualitativa (entrevistas, coleta de dados) na qual podemos identificar as necessidades dos usuários de um determinado portal.

	Para criação do persona será necessário realizar entrevista com 3 pessoas de cada grupo alvo (universitários, ativistas políticos, servidores públicos, etc).

\subsection{Paradigma e Técnica de Avaliação}

As medidas de usabilidade do sistema foram obtidas considerando-se três fatores: eficácia, eficiência e satifação de uso.

Neste experimento foi adotado como paradigma de avaliação o Teste de Usabilidade que consiste em avaliar o desempenho dos usuários na execução de tarefas cuidadosamente preparadas, tarefas estas dentro do escopo do sistema. Esse desempenho pode ser avaliado no quesito, número de erros e tempo de execução da tarefa.

A avaliação da usabilidade coletará tantos dados subjetivos em um cenário real como dados objetivos. Os dados subjetivos serão coletados através de opiniões dos participantes, comentários em relação ao uso do portal da participação social. Os dados objtivos consistem em medidas de tempo e desempenho dos participantes.

Para avaliar a usabilidade do portal participa.br serão utilizadas as técnicas:


\begin{table}[h]
\begin{tabular}{lllll}
\cline{1-2}
\multicolumn{1}{|l|}{\textbf{Técnica}}                & \multicolumn{1}{l|}{\textbf{Descrição}}                                                                                                                                    &  &  &  \\ \cline{1-2}
\multicolumn{1}{|l|}{\textbf{Observar Usuarios}}      & \multicolumn{1}{l|}{Um observador irá registrar o tempo gasto por cada participante para concluir o estudo de caso, avaliar a ferramenta e se necessitou de alguma ajuda.} &  &  &  \\ \cline{1-2}
\multicolumn{1}{|l|}{\textbf{Perguntar aos usuários}} & \multicolumn{1}{l|}{O questionário ASQ e PSSUQ de satisfação dos usuários será utilizado para coletar as opiniões dos participantes.}                                      &  &  &  \\ \cline{1-2}
                                                      &                                                                                                                                                                            &  &  & 
\end{tabular}
\end{table}



\subsection{Instrumentos para coletas de dados}

	Os intrumentos de coletas de informações utilizados são dois questionários que são amplamentes utilizados para medir a satisfação do usuário com produtos interativos e fornecem medidas padronizadas. São eles o After-Scenario Questionnaire (ASQ) \footnote{ASQ: Proposto por Lewis} e o Post-Study System Usabiliy Questionnaire (PSSUQ). 

	Optamos por não utilizar um questionário de satisfação próprio, pois necessitaria validar antes de ser utilizado.  Utilizamos então questionários já validados, o que facilitaria em nossa análise dos resultados.

	O ASQ é destinado ao uso em testes de usabilidade baseados em cenários. Em nosso trabalho definimos 6 cenários de uso com o objetivo de medir a satisfação relativa a cada tarefa. Possui três itens que abordam os seguintes componentes de usabilidade: (1) facilidade de conclusão da tarefa, (2) tempo necessário para completar uma tarefa e,(3) a adequação das instruções ou materiais de apoio fornecidos.

	O PSSUQ é aplicado apois a conclusão de todos os cenários com o próposito de fornecer uma avaliação geral geral da usabilidade do sistema. pois permite uma avaliação de usabilidade mais ampla, podendo avaliar 4 fatores e usabilidade (satifação geral, utilidade do sistema, qualidade da interface e qualidade da informação). 

	O objetivo do teste de usabilidade é exibir os problemas de usabilidade por meio da voz dos usuários típicos. Como cada um dos usuários participantes do teste se comporta na realização das atividades.



\subsubsection{Heurísticas de Usabilidade}

	Segundo as pesquisa realizadas é necessário que antes que execute um teste de usabilidade seja feito uma avaliação da usabilidade por parte de especialistas de usabilidade. Através das heurísticas de Nielsen identificamos os problemas inerentes ao portal da participação social.

	%falta fazer

\subsubsection{Cenários para teste de Usabilidade}

	Foram levantadas algumas tarefas que devem ser executadas para a realização do teste de usabilidade

\begin{enumerate}
	\item Faça seu cadastro no portal participa.br e ative sua conta.
	\item Personalize o seu perfil inserindo uma foto, escolha 5 categorias de interesse.
	\item Localize e adicione Jônatas Medeiros de Mendonça à sua rede.
	\item Localize e ingresse na comunidade Participação Social. Informe a quantidade de membros.
	\item Localize a pessoa Henrique Parra Filho e infome a quantidade de amigos, nº de comunidades 
\end{enumerate}

\subsubsection{Questões sobre o uso das funções do Portal da Participação Social}

	Levantamos algumas questões referentes ao uso das funcionalidades do portal da participação social	

\begin{enumerate}
	\item Quais funcionalidades da página inicial você já utilizou?
	\item Quais funcionalidades das comunidades você mais utiliza?
	\item Quais funcionalidades de administração você mais utiliza? 
	\item Quais funcionalidades das página de usuário você mais utiliza?
\end{enumerate}

\section{Planejamento}

\subsection{Definição de Hipóteses}

\textbf{Hipótese Nula (H0):} A média do grau de satisfação dos usuários que já utilizaram o portal seria maior do que quem nunca utilizou?

\textbf{Hipótese Alternativa (H1):} O grau de satisfação dos usuários que já tinha contato com o portal é diferente dos que nunca tiveram acesso.

% Obs: Definir melhor as hipóteses


\subsection{Seleção do Contexto}

\subsection{Seleção dos Indivíduos}

	Na primeira fase do experimento serão escolhidos pessoas com diferentes perfis que trabalham na Presidência da República e que já utilizaram o Portal da Participação Social.Na segunda afse seriam escolhidas pessoas que nunca utilizaram o portal.

\subsection{Variáveis}
 
a. Independentes

São variáveis que podem ser manipuladas no estudo experimental. É a “causa, antecedente, origem de um fenômeno, processo que constitui o objeto de estudo”.(Carrasco)

%Colocar tabela

A interface do Portal da Participação Social
Questionários de usabilidade.
Questionários de Perfil do usuário

%colocar tabela

b. Dependentes

É o efeito, consequência o resultado observado da influência das variávies independentes (Carrasco).

% Colocar tabela

\subsection{Recursos}

\begin{itemize}
\item Estação de trabalho para cada participante.
\item Navegador de Internet
\item Questionário para a avaliação da usabilidade. (Definir o questionário)
\item Software de Vídeo (Camtasia - versão trial) ou outro.
\end{itemize}

\subsection{Validade dos Resultados}


\begin{table}[h]
\begin{tabular}{|l|p{10cm}|l|l|}
\hline
\textbf{Ameaça}                                                                                                                                                      & \textbf{Tipo} & Descrição da ameaça                                                                                                                                             & Tratamento                                                                                                                   \\ \hline
\begin{tabular}[c]{@{}l@{}}O esforço por pessoas que já conhecem\\ o portal poderá ser maior do que com \\ pessoas que nunca teve contato com o portal.\end{tabular} & Externa       & \begin{tabular}[c]{@{}l@{}}Participantes que já tenha conhecimento \\ do portal terá uma maior facilidade de uso \\ pois já conhecem a ferramenta.\end{tabular} & \begin{tabular}[c]{@{}l@{}}Realizar também a pesquisa com \\ pessoas que nunca tiveram contato \\ com o portal.\end{tabular} \\ \hline
Questionário não preenchido                                                                                                                                          & Conclusão     & \begin{tabular}[c]{@{}l@{}}Participantes não preencherem todos \\ os itens dos questionários.\end{tabular}                                                      & \begin{tabular}[c]{@{}l@{}}Avisar aos participantes sobre \\ a importância de preencher todo\\ o questionário.\end{tabular}  \\ \hline
\begin{tabular}[c]{@{}l@{}}Quantidade de participantes insuficiente \\ para obter uma melhor amostra dos resultados\end{tabular}                                     & Externa       & \begin{tabular}[c]{@{}l@{}}Amostra muito pequena para \\ análise dos dados.\end{tabular}                                                                        & \begin{tabular}[c]{@{}l@{}}Realizar outro teste com \\ pessoas de diferentes lugares\end{tabular}                            \\ \hline
\end{tabular}
\end{table}

\section{Procedimentos para a execução}

\begin{itemize}

\item Para a execução do experimento serão testados alguns cenários de teste na qual os participantes devem executar um a um. Todos irão testar os mesmos cenários.
\item O estudo se inicia com a leitura da descrição do estudo de caso e como será a agenda de atividades. Serão explicados os cenários que cada um irá executar.
\item Após o período de exploração do portal e finalizada o estudo de caso (cerca de 30 min), os participantes devem responder o questionário geral.
Enquanto o participante realiza as atividades, um observador registra se o participante completou os cenários sem assistência e produziu a saída completa do caso de uso.
\item No final os participantes preenche um formulário de feedback.

\end{itemize}

\section{Avaliação dos Resultados}

\subsection{Plano de Avaliação}

\begin{table}[h]
\begin{tabular}{|l|l|}
\hline
\textbf{Técnica}               & \textbf{Descrição}                                                                                                                                                                                                 \\ \hline
Avaliação da Ferramenta        & Aplicação do questionário de usabilidade para o portal participa.br                                                                                                                                                \\ \hline
Registro de ocorrências        & \begin{tabular}[c]{@{}l@{}}Durante o experimento, um observador irá registrar todas as \\ ocorrências referentes à avaliação das ferramentas.\end{tabular}                                                         \\ \hline
Avaliação do Experimento       & \begin{tabular}[c]{@{}l@{}}Ao final da avaliação das ferramentas os participantes irão preencher \\ um questionário geral, avaliando o andamento do experiment\end{tabular}                                        \\ \hline
Relatório de análise dos dados & \begin{tabular}[c]{@{}l@{}}No final do estudo será feito um relatório com a análise dos dados\\ e lições aprendidas no que diz respeito à atuação da sua equipe \\ durante a execução do experimento.\end{tabular} \\ \hline
\end{tabular}
\end{table}

A avaliação dos resultados do experimento deve considerar o uso de técnicas estatísticas para analisar os dados e responder as questões referentes ao objetivo específico estabelecido no planejamento deste estudo.



