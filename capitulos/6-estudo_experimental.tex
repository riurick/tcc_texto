\chapter{Estudo de Caso}

%Neste capítulo será apresentado o projeto de estudo de caso.

Neste capítulo será apresentado o estudo de caso sobre a utilização de práticas de usabilidade no desenvolvimento do Portal do Software Público onde será definido um protocolo de como essas práticas podem ser inseriridas nesse contexto específico.


%avaliação de usabilidade a partir dos cenários de testes durante a colaboração com a plataforma livre Noosfero, explicitando os resultados obtidos.
%


\section{Definição}
Neste trabalho foi apresentado um estudo teórico relacionado à integração das técnicas de usabilidade ao longo do ciclo de vida ágil de desenvolvimento de software livre. Além de um estudo dos testes de softwares existentes e como podem impactar a usabilidade de um sistema.

% Inserir o escopo do estudo.

Assim, a proposta deste trabalho consiste na investigação, coleta, análise e discussão dos resultados de dados na adoção de práticas de usabilidade no desenvolvimento de software empírico.

\subsection{Objetivo Global}

O objetivo global desse estudo é analisar o impacto da adoção de práticas e técnicas de usabilidade no processo de desenvolvimento de software empírico.

%TODO Verificar se precisa de subseções
%TODO Repetição das palavras software empírico -verificar

\subsection{Objetivo de Medição}

Tendo em vista a utilização de várias técnicas de usabilidade e de testes de software em um projeto de desenvolvimento de software empírico, nossos objetivos de medição foram definidos à fim de avaliar:

\begin{enumerate}
\item Impacto da adoção de técnicas de usabilidade em um contexto específico de desenvolvimento de software.
\item A influência de práticas de BDD e TDD na usabilidade de um software. 
\end{enumerate}

A partir dos objetivos de medição estabelecidos, foram definidas questões de pesquisa de forma a entender se os objetivos específicos foram alcançados.

%Verificar uma forma melhor de escrever as questões
\begin{enumerate}
\item A utilização de práticas de usabilidade no ciclo de vida de desenvolvimento de software apresentaram melhores resultados no teste de aceitação?
\item O processo de desenvolvimento utilizando práticas de BDD e TDD apresentam melhores resultados nos testes de usabilidade?
\end{enumerate}

%Definir melhor isso
%TODO Definir métricas?

\subsection{Objeto de Estudo: Portal do Software Público Brasileiro - SPB}

O Portal do Software Público Brasileiro consiste em um sistema que permite o compartilhamento de softwares e faz parte da política de software livre no setor público.
O SPB será o objeto de estudo deste trabalho, assim como o processo de colaboração com o SPB, que é realizado no LAPPIS da UnB.

\section{Planejamento}

Nessa seção são apresentados a seleção dos contextos, a formualação das hipóteses, a seleção das variáveis, a seleção dos participantes.


\subsection{Definição de Hipóteses}


\subsection{Seleção do Contexto}

O contexto selecionado para o estudo de caso é o desenveolvimento do novo portal do software público, que vem sendo desenvolvido...

%TODO Qual o objetivo do sistema
%TODO Explicar o ambiente na qual está sendo inserido o estudo (lappis, ministerio, etc)? A linguagem, métodos, ferramentas nas quais estão utilizando.

\subsection{Seleção dos Indivíduos}

%TODO Falar quem são as pessoas envolvidas no projeto. Experiencia dos envolvidos em métodos ágeis e em usabilidade.
%TODO Definição de papeis


\subsection{Visão Geral do Projeto}

O processo de colaboração com o SPB baseia-se no desenvolvimento empírico e nas metodologias ágeis, como o scrum e o XP (Extreme Programming). O desenvolvimento de testes automatizados é intrínseco ao processo de colaboração com o SPB. A partir destes pontos mencionados, buscamos evoluir a forma com que o processo de colaboraçao com o SPB lida com problemas de usabilidade.
%Como funciona as reuniões e o que eles adotam
%Explicar a dinâmica de trabalho, a equipe, reuniões
%verificar aonde vai entrar essa seção. Talvex dentro do contexto?

\subsection{Metodologia}
%TODO não sei se aplica nesse item (seria o protocolo)
%Melhorar a escrita

	Seguindo as pesquisas realizadas sobre as técnicas de usabilidade e as metodologias existentes propostas por vários autores sobre a integração de tais técnicas em um contexto de software empírico, adotamos para este estudo de caso o seguinte protocolo que deve ser executado nas sprints de desenvolvimento do Portal do Software Público:
	
	
\begin{enumerate}

\item Análise de Usuários
	
	No ínicio de qualquer projeto é importante conhecer quem são os usuários que irão utilizar o sistema à ser desenvolvido. Para isso existe algumas técnicas que foram criadas para conhecer melhor quem são os usuários:
	
	\begin{itemize}
		\item Persona
		\item Roteiros		
	\end{itemize}
		% acho que é importante colocar um exemplo aqui ou apenas colocar o que será feito no estudo de caso em resultados.?
		%A metodologia XPU propõe a criação de Personas e Roteiros 
		
\item Análise do contexto de uso

		As Personas e os Roteiros geram as Histórias de usabilidade.

\item Definição de Requisitos e Metas de usabilidade

	Criação de Benchmarks pelos projetistas de interação e usuários para servir como balizadores para avaliar a qualidade da usabilidade que está sendo entregue ao final de cada iteração.
	Baseado nos atributos de usabilidade são estabelecidos instrumentos de medida para se obter valores quantitativos 	para cada atributo.
	Esses benchmarks são utilizados para planejar a avaliação de usabilidade e para realizar os testes que irão compor cada avaliação. % esse paragrafo deve ir pro capitulo de usabilidade onde fala de benchmarks.
	
		
\item Planejamento de Usabilidade

	Estimar os recursos relativos à usabilidade que serão utilizados ao longo do desenvolvimento. Previsões de tempo e custo no planejamento da release.
	 No planejamento da Release do projeto deve-se pensar também em estimativas de tamanho para as histórias de usabilidade e para o planejamento das avaliações de usabilidade.

\item Avaliação da usabilidade

	Ao longo do ciclo de vida diversas avaliações podem ser realizadas:
	
		
	\begin{itemize}
		\item Avaliação Heurística
		
		Antes de executar um teste de usabilidade é importante primeiro fazer uma avaliação heurística para identificar possíveis problemas que possam ser encontrados pelos usuários.
		
		\item Testes com usuários
		
		Em nosso estudo vamos propor que ao final de cada release seja feito um teste de usabilidade com 5 usuários, como é proposto por Nielsen.
				
	\end{itemize} 
	.
	
\item Controle de Usabilidade
	
	Ao final de cada interação a velocidade do projeto é avaliada. Deve-se acompanhar as tarefas relativas à usabilidade comparando o esforço previsto com o realizado.
	
\end{enumerate}
Esse protocolo foi extraído do estudo de várias metodologias criadas para a integração das técnicas de usabilidade no contexto de desenvolvimento ágil de software, são elas XPu e XPlus.
 
 
 %TODO Obs: No TCC 1 tinhamos escrito um protocolo de como seria feito á avaliação de usabilidade, algumas coisas podem ser utilizadas e readaptadas para o nosso contexto atual e inserindo as melhorias encontradas por outros métodos.
 
\section{Execução}

%Especificar as datas, as atividades realizadas.

O estudo de caso

\section{Coleta de Dados}
%TODO Explicar como será coletado os dados

\section{Análise e Interpretação dos Resultados}

Esta seção apresenta a discussão e a interpretação dos resultados observados durante a execução do estudo de caso descrito.

\subsection{Análise dos Dados}

Baseado nos objetivos do estudo de caso, os dados a serem levantados dizem respeito à ???

\subsection{Análise dos Resultados de Medição}

Analisa os resultados alcançados no estudo de caso, baseado ...

\subsection{Verificação das Hipóteses}

Esta subseção analisa os resultados alcançados no estudo de caso, baseado nas hipóteses levantadas e apresentadas no ínicio desta seção........


\section{Considerações finais}







