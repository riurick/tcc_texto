\chapter{Estudo de Caso}

Neste capítulo, apresentaremos osestudo de caso sobre a avaliação de usabilidade a partir dos cenários de testes durante a colaboração com a plataforma livre Noosfero, explicitando os resultados obtidos.
%

\section{Objeto de Estudo: Portal do Software Público Brasileiro - SPB}

O Portal do Software Público Brasileiro consiste em um sistema que permite o compartilhamento de softwares e faz parte da política de software livre no setor público.
O SPB será o objeto de estudo deste trabalho, assim como o processo de colaboração com o SPB, que é realizado no LAPPIS da UnB.

\subsection{Visão Geral do Projeto}

O processo de colaboração com o SPB baseia-se no desenvolvimento empírico e nas metodologias ágeis, como o scrum e o XP (Extreme Programming). O desenvolvimento de testes automatizados é intrínseco ao processo de colaboração com o SPB. A partir destes pontos mencionados, buscamos evoluir a forma com que o processo de colaboraçao com o SPB lida com problemas de usabilidade.
%Como funciona as reuniões e o que eles adotam
%Explicar a dinâmica de trabalho, a equipe, reuniões



\section{Coleta de Dados}
%TODO Explicar como será coletado os dados

\section{Diagnóstico e Prognóstico}

\section{Proposição de Solução e Tomada de Decisão}

\subsection{Avaliação dos Resultados}

\section{Considerações finais}







