\chapter{Estudo de Caso}

Neste capítulo será apresentado o estudo de caso sobre a utilização de práticas de usabilidade e testes automatizados no desenvolvimento do Portal do Software Público, baseado na plataforma Noosfero, onde será definido um guia de como essas práticas podem ser inseridas nesse contexto específico.

\section{Objeto de Estudo: Portal do Software Público Brasileiro - SPB}

O Portal do Software Público Brasileiro consiste em um sistema que permite o compartilhamento de softwares e faz parte da política de software livre no setor público.
O SPB será o objeto de estudo de caso, assim como o processo de colaboração com o SPB, que é realizado no LAPPIS da UnB.

\subsection{Visão Geral do Projeto}

O processo de colaboração com o SPB baseia-se no desenvolvimento empírico e nas metodologias ágeis scrum e XP. O desenvolvimento de testes automatizados é intrínseco ao processo de desenvolvimento. A partir destes pontos mencionados, buscamos evoluir a forma com que o processo de colaboraçao com o SPB lida com problemas de usabilidade.
O processo de desenvolvimento é feito a partir de sprints de duas semanas, em que são realizadas reuniões com as equipes e reuniões de planejamento (Planning Poker) em cada equipe para definir as atividades de cada sprint.

%Explicar a dinâmica de trabalho, a equipe, reuniões
%verificar aonde vai entrar essa seção. Talvex dentro do contexto?

\section{Execução do estudo de caso}

\subsection{Problemas encontrados}

No acompanhamento do projeto podemos identificar inicialmente os seguintes problemas relacionados à usabilidade no desenvolvimento de software:

%descrever problemas

 
\section{Execução}

%\subsection{Práticas adotadas pela equipe}

\textbf{Questionário Online}

Inicialmente à equipe de design realizou um planejamento para definir o instrumento de pesquisa dos usuários. Foi definido que o principal objetivo do estudo seria medir a percepção dos usuários quanto a qualidade de uso do portal atual sob uma perspectiva global. Nesse sentido foi proposto uma abordagem de pesquisa de levantamento aplicada e quantitativa. A qualidade de uso foi compreendida a partir dos parâmetros propostos pela ISO 9241-10 a 17, como a qualidade da interface de proporcionar eficácia, eficiência e satisfação no uso do portal (APPOLINÁRIO, 2004). 

Na definição do delineamento da pesquisa foi selecionada a técnica de aplicação de questionário on-line para se produzir os dados de opinião dos participantes. Entende-se que a dimensão atitudinal dos usuários do portal com relação a sua experiência de uso só podem ser obtidos por meio do registro de suas verbalizações (MARTINS E THEOPHILO, 2009). No mesmo sentido, a avaliação proposta não permite uma compreensão direta e completa a cerca da usabilidade do portal em si, compreendendo-se que a efetividade do portal só pode ser analisada 
diretamente por meio do uso de outras técnicas de pesquisa, como as observações globais ou sistemáticas da interação entre os participantes e a sua interface % (MARTINS e THEÓPHILO, 2009; ABRAHÃO et al., 2009).

\textbf{Protótipos}
	A equipe utiliza-se de protótipos para determinar os cenários de uso do software, baseando-se nos requisitos definidos pelos clientes. Abaixo estão os protótipos de cadastro de usuário e cadastro de software:

	\begin{figure}[h!]
    	\centering
    	\includegraphics[keepaspectratio=true,scale=0.32]
      		{figuras/CadastroEdicaoUser.eps}
    	\caption{Protótipos de cadastro de usuário}
    	\label{cadastro_user}
	\end{figure}

	\begin{figure}[h!]
    	\centering
    	\includegraphics[keepaspectratio=true,scale=0.25]
      		{figuras/CadastroEdicaoSoftware.eps}
    	\caption{Protótipos de cadastro de software}
    	\label{cadastro_software}
	\end{figure}

\newpage

\textbf{Cenários de Uso}
	Para cada funcionalidade desenvolvida é determinado um cenário de uso, base para a implementação dos testes de aceitação e consequentemente o desenvolvimento da funcionalidade propriamente dita.
	Durante a primeira release (release 0) foram desenvolvidas algumas histórias, dentre estas, a história de ``Cadastro de Usuário", que possui os seguintes cenários de sucesso:

	\textbf{Cenário 01:} Cadastro com sucesso de apenas campos obrigatórios
	\textbf{[Dado]} que não existe nenhum usuário com o nome de usuário josesilva 
	\textbf{[Quando]} eu clicar em cadastrar novo usuário
	\textbf{[E]} eu preencho os seguintes campos: 
  		nome de usuário: josesilva
  		e-mail: jose@gmail.com
  		senha: 123456
  		confirmação da senha: 123456
  		nome completo: José da Silva 
  		país: Brasil
  		estado: Distrito Federal
  		cidade: Brasília
	\textbf{[E]} eu clico em cadastrar
	\textbf{[Então]} eu recebo uma confirmação de cadastro realizado com sucesso

	\textbf{Cenário 02:} Cadastro com sucesso de apenas campos obrigatórios de usuário governamental
	\textbf{[Dado]} que não existe nenhum usuário com o nome de usuário josesilva 
	\textbf{[Quando]} eu clicar em cadastrar novo usuário
	\textbf{[E]} eu preencho os seguintes campos: 
  		nome de usuário: josesilva
  		e-mail: jose@serpro.gov.br
  		e-mail secundário: jose@gmail.com
  		senha: 123456
		  confirmação da senha: 123456
		  nome completo: José da Silva 
		  cargo: analista de TI
		  país: Brasil
		  estado: Distrito Federal
		  cidade: Brasília
	\textbf{[E]} eu seleciono ``SERPRO" como instituição
	\textbf{[E]} eu seleciono ``????" como unidade  
	\textbf{[E]} eu clico em cadastrar
	\textbf{[Então]}eu recebo uma confirmação de cadastro realizado com sucesso

\textbf{Cenário 3:} Cadastro com sucesso com todos os campos preenchidos, mesmo não obrigatórios
	\textbf{[Dado]} que não existe um usuário cujo email primário ou email secundário é maria@gmail.com
	\textbf{[Quando]} eu clicar em cadastrar novo usuário
	\textbf{[E]} eu preencho os seguintes campos: 
  		nome de usuário: mariasilva
		  e-mail: maria@gmail.com
		  e-mail secundário: maria@yahoo.com
		  senha: 123456
		  confirmação da senha: 123456
		  nome completo: Maria da Silva
		  cargo: analista de TI
		  áreas de interesse: Engenharia de Software;
		  país: Brasil
		  estado: Distrito Federal
		  cidade: Brasília
	\textbf{[E]} eu seleciono ``Outro" como instituição 
	\textbf{[E]} eu clico em cadastrar
	\textbf{[Então]} eu recebo uma notificação de cadastro realizado com sucesso.

	Os cenários de falha ocorrem nas seguintes situações:
	\begin{itemize}
	\item Email proposto exisitir como email de outro usuário;
	\item Email secundário propostro exisitir como email de outro usuário;
	\item Email secundário ser um email governamental e ao email primário não ser um email governamental;
	\item Não preenchimento de campos obrigatórios para usuário governamental 
	\end{itemize}
%Especificar as datas, as atividades realizadas.


Outra funcionalidade desenvolvida foi a história chamada ``Manter Instituição", que possui os seguintes cenários:

\textbf{Cenário 01:} Cadastro de nova instituição com sucesso
\textbf{[Dado]} que eu estou na página de cadastro de usuário
\textbf{[E]} que a seguinte instituição não existe:
  	nome: Ministério do Planejamento, Orçamento e Gestão
  	sigla: MP
 	poder: executivo
 	esfera: federal
  	tipo: pública
  	cnpj: 00.489.828/0002-36
\textbf{[Quando]} eu clicar em ``Cadastrar nova instituição" 
\textbf{[E]} eu preencher os seguintes campos:
  	sigla: MP
  	poder: executivo
  	esfera: federal
  	tipo: pública
  	cnpj: 00.489.828/0002-36
\textbf{[Então]} eu devo visualizar a mensagem ``Instituição cadastrada com sucesso!"

\textbf{Cenário 02:} Busca de instituição inexistente
\textbf{[Dado]} que eu estou na página de cadastro de usuário
\textbf{[E]} que a seguinte instituição não existe:
  nome: Ministério do Planejamento, Orçamento e Gestão
  sigla: MP
  poder: executivo
  esfera: federal
  tipo: pública
  cnpj: 00.489.828/0002-36
\textbf{[Quando]} eu buscar MP
\textbf{[Então]} eu devo visualizar a mensagem ``Instituição não cadastrada" 
\textbf{[E]}eu devo visualizar a opção de cadastrar nova instituição

Estes cenários apresentam alguns problemas de usabilidade analisando-os de acordo com as heurísticas de Nielsen, e após o processo de desenvolvimento ter uma melhoria na visão de usabilidade após a incorporação de profissionais da área, os cenários apresentaram melhoras em relação às heurísticas de Nielsen. 

Durante a segunda release (release 1) as histórias de ``Cadastro de Usuário", e ``Manter Instituição" foram desenvolvidas novamente com os seguintes cenários:

\textbf{Cenário 01:} Cadastro com sucesso de apenas campos obrigatórios
	\textbf{[Dado]} que não existe nenhum usuário com o nome de usuário josesilva 
	\textbf{[Quando]} eu clicar em ``Cadastre-se"
	\textbf{[E]} eu preencho os seguintes campos: 
  		primeiro nome: José
  		ultimo nome: Silva
  		endereço de e-mail: jose@gmail.com
  		usuário: josesilva
  		
	\textbf{[E]} eu clico em ``Cadastre-se"
	\textbf{[Então]} eu recebo uma confirmação de cadastro realizado com sucesso, com a seguinte mensgamte: 
	``Você deve se logar para seu perfil. Perfis não validados serão deletados em 24h."


Para dar continuidade a este processo este estudo de caso avaliou os cenários estabelecidos e suas evoluções, verificando e propondo melhorias.

Durante a segunda release (release 2) a história de ``Novo Software" foi desenvolvida com os seguintes cenários:

\textbf{Cenário 01:} Novo Software
	\textbf{[Dado]} que não existe nenhum software com o localhost ``software"
	\textbf{[Quando]} eu clicar em ``Novo Software"
	\textbf{[E]} eu preencho os seguintes campos: 
  		localhost: software 
  		finalidade: Finalidade do software
  		licenca: licença
  		
  		
	\textbf{[E]} eu clico em ``Salvar"
	\textbf{[Então]} eu recebo uma confirmação de cadastro realizado com sucesso, e encontro a pagina de edição de software


Outros cenários de edição de software são: ``Informações de Comunidade" e ``Informações de Software".

\textbf{Cenário 02:} Informações de Comunidade
	\textbf{[Dado]} que ``software" está cadastrado
	\textbf{[Quando]} eu clicar em ``Cadastre-se"
	\textbf{[E]} eu preencho os seguintes campos: 
  		descricao: Descricao do software 
  		tags: software
  		categorias: categoria1
 
	\textbf{[E]} eu clico em ``Salvar"
	\textbf{[Então]} eu recebo uma confirmação de cadastro salvo com sucesso


\textbf{Cenário 03:} Informações de Software
	\textbf{[Dado]} que ``software" está cadastrado e estou em Edição de software
	\textbf{[Quando]} eu clicar em ``Especifico"
	\textbf{[E]} eu preencho os seguintes campos: 
  		Sigla: teste 
  		sistema operacional: teste os
  		funcioanalidades: testes
  		categorias: categoria1
 	
 	\textbf{[E]} eu clico em ``Nova Biblioteca"
 	\textbf{[E]} eu preencho os seguintes campos: 

 		nome: teste
 		versao: teste
 		licenca: teste

 	\textbf{[E]} eu clico em ``Novo Sistema Operacional"
 	\textbf{[E]} eu preencho os seguintes campos: 

 		nome: Debian
 		versão: teste
 	\textbf{[E]} eu clico em ``Nova linguagem"
 	\textbf{[E]} eu preencho os seguintes campos: 

 		nome: C++
 		versão: teste
 		sistema operacional: Debian
 	\textbf{[E]} eu clico em ``Novo Banco de Dados"
 	\textbf{[E]} eu preencho os seguintes campos: 

 		nome: apache
 		versão: teste
 		sistema operacional: Debian
	\textbf{[E]} eu clico em ``Salvar"
	\textbf{[Então]} eu recebo uma confirmação de cadastro salvo com sucesso


\section{Análise e Interpretação dos Resultados}

Esta seção apresenta a discussão e a interpretação dos resultados observados durante a execução do estudo de caso descrito.

Analisando os protótipos e os cenários de uso desenvolvidos, de acordo com as heurísitcas de Nielsem, encontramos alguns problemas de usabilidade e propomos melhorias a serem feitas e possivelmente verificadas durante os testes de aceitação.

\subsection{Análise dos Dados}

Baseado nos objetivos do estudo de caso, os dados a serem levantados dizem respeito à ???

\subsection{Análise dos Resultados de Medição}

Analisa os resultados alcançados no estudo de caso, baseado ...

\subsection{Verificação}

Esta subseção analisa os resultados alcançados no estudo de caso, baseado nas hipóteses levantadas e apresentadas no ínicio desta seção........


\section{Considerações finais do capítulo}







