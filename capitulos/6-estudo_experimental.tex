\chapter{Estudo de Caso}

Neste capítulo será apresentado o estudo de caso sobre a utilização de práticas de usabilidade e testes automatizados no desenvolvimento do Portal do Software Público, baseado na plataforma Noosfero, onde será definido um guia de como essas práticas podem ser inseridas nesse contexto específico.

\section{Objeto de Estudo: Portal do Software Público Brasileiro - SPB}

O Portal do Software Público Brasileiro consiste em um sistema que permite o compartilhamento de softwares e faz parte da política de software livre no setor público.
O SPB será o objeto de estudo de caso, assim como o processo de colaboração com o SPB, que é realizado no LAPPIS da UnB.

\subsection{Visão Geral do Projeto}

O processo de colaboração com o SPB baseia-se no desenvolvimento empírico e nas metodologias ágeis scrum e XP. O desenvolvimento de testes automatizados é intrínseco ao processo de desenvolvimento. A partir destes pontos mencionados, buscamos evoluir a forma com que o processo de colaboraçao com o SPB lida com problemas de usabilidade.
O processo de desenvolvimento é feito a partir de sprints de duas semanas, em que são realizadas reuniões com as equipes e reuniões de planejamento (Planning Poker) em cada equipe para definir as atividades de cada sprint.

%Explicar a dinâmica de trabalho, a equipe, reuniões
%verificar aonde vai entrar essa seção. Talvex dentro do contexto?


 
\section{Execução}

\subsection{Práticas adotadas}


No acompanhamento do projeto podemos identificar inicialmente os seguintes problemas relacionados à usabilidade no desenvolvimento de software:

\textbf{Questionário Online}

Inicialmente à equipe de design realizou um planejamento para definir o instrumento de pesquisa dos usuários. Foi definido que o principal objetivo do estudo seria medir a percepção dos usuários quanto a qualidade de uso do portal atual sob uma perspectiva global. Nesse sentido foi proposto uma abordagem de pesquisa de levantamento aplicada e quantitativa. A qualidade de uso foi compreendida a partir dos parâmetros propostos pela ISO 9241-10 a 17, como a qualidade da interface de proporcionar eficácia, eficiência e satisfação no uso do portal (APPOLINÁRIO, 2004). 

Na definição do delineamento da pesquisa foi selecionada a técnica de aplicação de questionário on-line para se produzir os dados de opinião dos participantes. Entende-se que a dimensão atitudinal dos usuários do portal com relação a sua experiência de uso só podem ser obtidos por meio do registro de suas verbalizações (MARTINS E THEOPHILO, 2009). No mesmo sentido, a avaliação proposta não permite uma compreensão direta e completa a cerca da usabilidade do portal em si, compreendendo-se que a efetividade do portal só pode ser analisada 
diretamente por meio do uso de outras técnicas de pesquisa, como as observações globais ou sistemáticas da interação entre os participantes e a sua interface % (MARTINS e THEÓPHILO, 2009; ABRAHÃO et al., 2009).

\textbf{Protótipos}
	A equipe utiliza-se de protótipos para determinar os cenários de uso do software, baseando-se nos requisitos definidos pelos clientes. Abaixo estão os protótipos de cadastro de usuário e cadastro de software:

	\begin{figure}[h!]
    	\centering
    	\includegraphics[keepaspectratio=true,scale=0.32]
      		{figuras/CadastroEdicaoUser.eps}
    	\caption{Protótipos de cadastro de usuário}
    	\label{cadastro_user}
	\end{figure}

	\begin{figure}[h!]
    	\centering
    	\includegraphics[keepaspectratio=true,scale=0.25]
      		{figuras/CadastroEdicaoSoftware.eps}
    	\caption{Protótipos de cadastro de software}
    	\label{cadastro_software}
	\end{figure}

\newpage

\textbf{Cenários de Uso}
	Para cada funcionalidade desenvolvida é determinado um cenário de uso, base para a implementação dos testes de aceitação e consequentemente o desenvolvimento da funcionalidade propriamente dita.
	Durante a primeira release (release 0) foram desenvolvidas algumas histórias, dentre estas, a história de ``Cadastro de Usuário'', que possui os seguintes cenários de sucesso:

	\begin{itemize}
	\item\textbf{Cenário 01:} Cadastro com sucesso de apenas campos obrigatórios

	\textbf{[Dado]} que não existe nenhum usuário com o nome de usuário ``josesilva''

	\textbf{[Quando]} eu clicar em cadastrar novo usuário

	\textbf{[E]} eu preencho os seguintes campos: 

  		\subitem nome de usuário: ``josesilva''

  		\subitem e-mail: ``jose@gmail.com''

  		\subitem senha: ``123456''

  		\subitem confirmação da senha: ``123456''

  		\subitem nome completo: ``José da Silva''

  		\subitem país: ``Brasil''

  		\subitem estado: ``Distrito Federal''

  		\subitem cidade: ``Brasília''

	\textbf{[E]} eu clico em cadastrar

	\textbf{[Então]} eu recebo uma confirmação de cadastro realizado com sucesso


	\item\textbf{Cenário 02:} Cadastro com sucesso de apenas campos obrigatórios de usuário governamental
	
	\textbf{[Dado]} que não existe nenhum usuário com o nome de usuário ``josesilva''
	
	\textbf{[Quando]} eu clicar em cadastrar novo usuário
	
	\textbf{[E]} eu preencho os seguintes campos: 
  		\subitem nome de usuário: ``josesilva''

  		\subitem e-mail: ``jose@serpro.gov.br''

  		\subitem e-mail secundário: ``jose@gmail.com''

  		\subitem senha: ``123456''

		\subitem confirmação da senha: ``123456''

		\subitem nome completo: ``José da Silva''

		\subitem cargo: ``analista de TI''

		\subitem país: ``Brasil''

		 \subitem estado: ``Distrito Federal''

		\subitem cidade: ``Brasília''

	\textbf{[E]} eu seleciono ``SERPRO'' como instituição

	\textbf{[E]} eu seleciono ``????'' como unidade  

	\textbf{[E]} eu clico em cadastrar

	\textbf{[Então]}eu recebo uma confirmação de cadastro realizado com sucesso

\item\textbf{Cenário 3:} Cadastro com sucesso com todos os campos preenchidos, mesmo não obrigatórios

	\textbf{[Dado]} que não existe um usuário cujo email primário ou email secundário é ``maria@gmail.com''

	\textbf{[Quando]} eu clicar em cadastrar novo usuário

	\textbf{[E]} eu preencho os seguintes campos: 

  		\subitem nome de usuário: ``mariasilva''

		  \subitem e-mail: ``maria@gmail.com''

		  \subitem e-mail secundário: ``maria@yahoo.com''

		  \subitem senha: ``123456''

		  \subitem confirmação da senha: ``123456''

		  \subitem nome completo: ``Maria da Silva''

		  \subitem cargo: ``analista de TI''

		  \subitem áreas de interesse: ``Engenharia de Software''

		  \subitem país: ``Brasil''

		  \subitem estado: ``Distrito Federal''

		  \subitem cidade: ``Brasília''

	\textbf{[E]} eu seleciono ``Outro'' como instituição 

	\textbf{[E]} eu clico em cadastrar

	\textbf{[Então]} eu recebo uma notificação de cadastro realizado com sucesso.
	\end{itemize}

	Os cenários de falha ocorrem nas seguintes situações:
	\begin{itemize}
	\item Email proposto exisitir como email de outro usuário;
	\item Email secundário propostro exisitir como email de outro usuário;
	\item Email secundário ser um email governamental e ao email primário não ser um email governamental;
	\item Não preenchimento de campos obrigatórios para usuário governamental 
	\end{itemize}
%Especificar as datas, as atividades realizadas.


Outra funcionalidade desenvolvida foi a história chamada ``Manter Instituição'', que possui os seguintes cenários:
\begin{itemize}
\item\textbf{Cenário 01:} Cadastro de nova instituição com sucesso

\textbf{[Dado]} que eu estou na página de cadastro de usuário

\textbf{[E]} que a seguinte instituição não existe:

  	\subitem nome: ``Ministério do Planejamento, Orçamento e Gestão''

  	\subitem sigla: ``MP''

 	\subitem poder: ``executivo''

 	\subitem esfera: ``federal''

  	\subitem tipo: ``pública''

  	\subitem cnpj: ``00.489.828/0002-36''

\textbf{[Quando]} eu clicar em ``Cadastrar nova instituição''

\textbf{[E]} eu preencher os seguintes campos:

  	\subitem sigla: ``MP''

  	\subitem poder: ``executivo''

  	\subitem esfera: ``federal''

  	\subitem tipo: ``pública''

  	\subitem cnpj: ``00.489.828/0002-36''

\textbf{[Então]} eu devo visualizar a mensagem ``Instituição cadastrada com sucesso!''

\item\textbf{Cenário 02:} Busca de instituição inexistente

\textbf{[Dado]} que eu estou na página de cadastro de usuário

\textbf{[E]} que a seguinte instituição não existe:

 \subitem nome: ``Ministério do Planejamento, Orçamento e Gestão''

  \subitem sigla: ``MP''

  \subitem poder: ``executivo''

  \subitem esfera: ``federal''

  \subitem tipo: ``pública''
  
  \subitem cnpj: ``00.489.828/0002-36''

\textbf{[Quando]} eu buscar MP

\textbf{[Então]} eu devo visualizar a mensagem ``Instituição não cadastrada''

\textbf{[E]}eu devo visualizar a opção de cadastrar nova instituição
\end{itemize}

Estes cenários apresentam alguns problemas de usabilidade analisando-os de acordo com as heurísticas de Nielsen, e após o processo de desenvolvimento ter uma melhoria na visão de usabilidade após a incorporação de profissionais da área, os cenários apresentaram melhoras em relação às heurísticas de Nielsen. 

Durante a segunda release (release 1) as histórias de ``Cadastro de Usuário'', e ``Manter Instituição'' foram desenvolvidas novamente com os seguintes cenários:

\begin{itemize}
\item\textbf{Cenário 01:} Cadastro com sucesso de apenas campos obrigatórios

	\textbf{[Dado]} que não existe nenhum usuário com o nome de usuário ``josesilva''

	\textbf{[Quando]} eu clicar em ``Cadastre-se''

	\textbf{[E]} eu preencho os seguintes campos: 

  		\subitem primeiro nome: ``José''

  		\subitem ultimo nome: ``Silva''

  		\subitem endereço de e-mail: ``jose@gmail.com''

  		\subitem usuário: ``josesilva''
  		
	\textbf{[E]} eu clico em ``Cadastre-se''

	\textbf{[Então]} eu recebo uma confirmação de cadastro realizado com sucesso, com a seguinte mensgamte: 
	``Você deve se logar para seu perfil. Perfis não validados serão deletados em 24h.''
\end{itemize}

Para dar continuidade a este processo este estudo de caso avaliou os cenários estabelecidos e suas evoluções, verificando e propondo melhorias.

Durante a segunda release (release 2) a história de ``Novo Software'' foi desenvolvida com os seguintes cenários:

\begin{itemize}
\item\textbf{Cenário 01:} Novo Software

	\textbf{[Dado]} que não existe nenhum software com o localhost ``software''

	\textbf{[Quando]} eu clicar em ``Novo Software''

	\textbf{[E]} eu preencho os seguintes campos: 

  		\subitem localhost: ``software''

  		\subitem finalidade: ``Finalidade do software''

  		\subitem licenca: ``licença''
  		
  		
	\textbf{[E]} eu clico em ``Salvar''

	\textbf{[Então]} eu recebo uma confirmação de cadastro realizado com sucesso, e encontro a pagina de edição de software


Outros cenários de edição de software são: ``Informações de Comunidade'' e ``Informações de Software''.

\item\textbf{Cenário 02:} Informações de Comunidade

	\textbf{[Dado]} que ``software'' está cadastrado

	\textbf{[Quando]} eu clicar em ``Cadastre-se''

	\textbf{[E]} eu preencho os seguintes campos: 

  		\subitem descricao: ``Descricao do software''

  		\subitem tags: ``software''

  		\subitem categorias: ``categoria1''
 
	\textbf{[E]} eu clico em ``Salvar''

	\textbf{[Então]} eu recebo uma confirmação de cadastro salvo com sucesso


\item\textbf{Cenário 03:} Informações de Software

	\textbf{[Dado]} que ``software'' está cadastrado e estou em Edição de software

	\textbf{[Quando]} eu clicar em ``Especifico''

	\textbf{[E]} eu preencho os seguintes campos: 

  		\subitem Sigla: ``teste''

  		\subitem sistema operacional: ``teste os''

  		\subitem funcioanalidades: ``testes''

  		\subitem categorias: ``categoria1''
 	
 	\textbf{[E]} eu clico em ``Nova Biblioteca''

 	\textbf{[E]} eu preencho os seguintes campos: 

 		\subitem nome: ``teste''

 		\subitem versao: ``teste''

 		\subitem icenca: ``teste''

 	\textbf{[E]} eu clico em ``Novo Sistema Operacional''

 	\textbf{[E]} eu preencho os seguintes campos: 

 		\subitem nome: ``Debian''

 		\subitem versão: ``teste''

 	\textbf{[E]} eu clico em ``Nova linguagem''

 	\textbf{[E]} eu preencho os seguintes campos: 

 		\subitem nome: ``C++''

 		\subitem versão: ``teste''

 		\subitem sistema operacional: ``Debian''

 	\textbf{[E]} eu clico em ``Novo Banco de Dados''

 	\textbf{[E]} eu preencho os seguintes campos: 

 		\subitem nome: ``apache''

 		\subitem versão: ``teste''

 		\subitem sistema operacional: ``Debian''

	\textbf{[E]} eu clico em ``Salvar''

	\textbf{[Então]} eu recebo uma confirmação de cadastro salvo com sucesso

\end{itemize}


\textbf{Testes de Aceitação}

Os testes de aceitação software público são responsáveis por verificar os seguintes fatores:

\begin{itemize}
	\item Capacidade de registrar e editar informações de usuário e instituição;
	\item Capacidade de registrar e editar informações de software;
	\item Capacidade de desativar usuário;
	\item Capacidade de desativar software;
\end{itemize}


Segue abaixo os testes desenvolvidos para e edição de software:

\textbf{Feature:} software registration
  As a user
  I want to create a new software
  So that I can have software communities on my network

  \textbf{Background:}
    \textbf{Given} ``MpogSoftwarePlugin'' plugin is enabled
    \textbf{And} SoftwareInfo has initial default values on database
    \textbf{And} I am logged in as admin
    \textbf{And} I go to /admin/plugins
    \textbf{And} I check ``MpogSoftwarePlugin''
    \textbf{And} I press ``Save changes''

  \textbf{Scenario:} Show library fields when click in New Library
    \textbf{Given} I go to admin_user's control panel
    \textbf{And} I follow ``Manage my groups''
    \textbf{And} I follow ``Create a new software''
    \textbf{And} I follow ``New Library''
    \textbf{Then} I should see ``Name''
    \textbf{Then} I should see ``Version''
    \textbf{Then} I should see ``License''

  
  \textbf{Scenario:} Show SoftwareLangue fields when click in New Language
    \textbf{Given} I go to admin_user's control panel
    \textbf{And} I follow ``Manage my groups''
    \textbf{And} I follow ``Create a new software''
    \textbf{And} I follow ``New language''
    \textbf{And} I should see ``3'' of this selector ``.software-language-table''
    \textbf{And} I follow ``Delete''
    \textbf{Then} I should see ``2'' of this selector ``.software-language-table''
    

 
  \textbf{Scenario:} Show databasefields when click in New database
    \textbf{Given} I go to admin_user's control panel
     \tetxbf{And} I follow ``Create a new software''
     \tetxbf{And} I follow ``Manage my groups''
     \tetxbf{And} I follow ``New Database''
     \tetxbf{And} I should see ``3'' of this selector ``.database-table''
     \tetxbf{And} I follow ``Delete''
    \textbf{Then} I should see ``2'' of this selector ``.database-table''
    #3 because one is always hidden

  
  \textbf{Scenario}: Delete software libraries
    \textbf{Given} I go to admin_user's control panel
    \textbf{And} I follow ``Manage my groups''
    \textbf{And} I follow ``Create a new software''
    \textbf{And} I follow ``New Library''
    \textbf{And} I should see ``2'' of this selector ``.library-table''
    \textbf{And} I follow ``Delete''
   \textbf{Then} I should see ``1'' of this selector ``.library-table''


\section{Análise e Interpretação dos Resultados}

Esta seção apresenta a discussão e a interpretação dos resultados observados durante a execução do estudo de caso descrito.

Analisando os protótipos e os cenários de uso desenvolvidos, de acordo com as heurísitcas de Nielsem, encontramos alguns problemas de usabilidade e propomos melhorias a serem feitas e possivelmente verificadas durante os testes de aceitação.

\subsection{Análise dos Dados}

A partir dos protótipos e dos cenários de uso desenvolvidos durante as releases 0 e 1, realizamos análises dos cenários em busca de problemas de usabilidade, com base nas heurísticas de Nielsem. A tabela abaixo descreve os problemas encontrados nas primeiras análises.

A partir do levantamento desses problemas foram propostas tarefas durante o planejamento de atividades da equipe de desenvolvimento, para que assim cada problema pudesse ser discutido e resolvido.



\begin{table}[d]
\begin{tabular}{|l|p{3cm}|p{6cm}|p{3cm}|l|}
\hline
\textbf{ID} & \textbf{Local} & \textbf{Descrição do Problema}                                                                                     & \textbf{Heurística Desobedecida} & \textbf{Criticidade} \\ \hline
1           & Protótipo - Novo Software                 & Ao escolher um nome para o software, não há evidência do que aconteceria caso o nome fosse igual a outro existente & Prevenção de erros               & Média                \\ \hline
2           & Protótipo - Novo Software                 & Não há informação para o usuário do que seria o ``Link''                                                             & Ajuda e documentação             & Média                \\ \hline
3           & Protótipo - Nova Comundiade               & Palavras em inglês e português                                                                                     & Liguagem Clara                   & Baixa                \\ \hline
4           & Protótipo - Nova Comundiade               & Falta de informação sobre as tags e as categorias do noosfero                                                      & Ajuda e documentação             & Baixa                \\ \hline
5           & Protótipo - Novo Software                 & Campos obrigatórios não são definidos                                                                              & Prevenção de erros               & Média                \\ \hline
6           & Protótipo - Novo Software    & Mensagens de ajuda ao preenchimentos dos campos                                                                    & Prevenção de erros               & Baixa                \\ \hline
7           & Protótipo - Nova Comundiade               & Botao de Save e Continue altera estrutura da pagina de software                                                    & Consistência                     & Baixa                \\ \hline
\end{tabular}
\end{table}



\begin{table}[d]
\begin{tabular}{|l|p{3cm}|p{6cm}|p{3cm}|l|}
\hline
\textbf{ID} & \textbf{Local} & \textbf{Descrição do Problema}                                                                                     & \textbf{Heurística Desobedecida} & \textbf{Criticidade} \\ \hline
1           & Cadastro de Usuário                 & Linguagens estrangeiras e avisos em outros idiomas & Diálogos simples  & Média                \\ \hline
2           & Cadastro de Usuário      & Botão de adicionar nova instituição não é claro para o usuário.  & Minimizar a sobrecarga de memória do usuário;           & Alta                \\ \hline
3           & Cadastro de Usuário               & Diferença entre botão adicionar nova instituição e criar nova instituição para o usuário  & Minimizar a sobrecarga de memória do usuário & Media                \\ \hline
4           & Cadastro de Instituição             & Seleção de País deve ter Brasil como default  & Diálogos simples e naturais    & Baixa                \\ \hline
5           & Cadastro de Instituição      & Opção de escolha do estado: Random button não posicionado e não funciona no primeiro clique. & Diálogos simples e naturais  & Baixa                \\ \hline
6           & Cadastro de Usuário  & Caso você não preencha o email secundario ele informa a seguinte mensagem: E-mail or secondary e-mail already taken & Boas mensagens de erros                & Média                \\ \hline
7           & Cadastro de Usuário  & Opção recaptcha só aparece na segunda vez & Consistência                     & Alta                \\ \hline
8           & Perfil de Usuário  & Ao clicar em hide no bloco de progresso de perfil ele some e não foi encontrada uma opção fácil para reverter a situação.
 & Prevenção de erros                     & Baixa                \\ \hline
9           & Cadastro de Usuário  & Caso se tenha uma infinidade de grupos, fica inviável a opção de checkbox. Ou será utilizada apenas para grupos mais ``conhecidos''? & Atalhos & Média                \\ \hline
10           & Cadastro de Usuário  & Mensagem de erro um pouco confusa: Usuário com este Usuário já existe.  & Consistência                     & Baixa                \\ \hline
\end{tabular}
\end{table}



 




\subsection{Análise dos Resultados de Medição}


Analisando os dados das primeiras releases, que comporam as primeiras rodadas de avalições pelas heurísticas temos em ``Cadastro de Usuário'':
\begin{itemize}
	\item 2 problemas de criticidade baixa;
	\item 4 problemas de criticidade média;
	\item 2 problemas de criticidade alta;
\end{itemize}

Já em ``Cadastro de Instiuição'', temos:
\begin{itemize}
	\item 2 problemas de criticidade baixa;
	\item 0 problemas de criticidade média;
	\item 0 problemas de criticidade alta;
\end{itemize}

Quanto à história de ``Cadastro de  Software'', temos:
\begin{itemize}
	\item 4 problemas de criticidade baixa;
	\item 3 problemas de criticidade média;
	\item 0 problemas de criticidade alta;
\end{itemize}

\subsection{Verificação}

Esta subseção analisa os verifica alcançados no estudo de caso, baseado nas hipóteses levantadas e apresentadas no ínicio desta seção. Podemos ver que a partir das avaliações de heurísticas nos protótipos e cenários diminuiu o numero de casos com problemas de criticidade média e alta. A release 0 (com as histórias de ``Cadastro de Usuário'' e ``Cadastro de Instituição'') teve 2 problemas de criticidade alta e 4 de criticidade média, já a release 1 (com a história de ``Cadastro de Software'') teve 3 problemas de criticidade média e nenhum de cirticidade alta, o número de problemas de criticidade baixa, se mantiveram em 4.



\section{Considerações finais do capítulo}







