\chapter{Estudo de Caso: Noosfero}
\label{noosfero}
%------------------------------------------------------------------------------%
Noosfero é uma plataforma desenvolvida pela Colivre (Cooperativa de Tecnologias
Livre), possui licença AGPL e é utilizado, principalmente em universidades públicas,
para desenvolvimento de redes colaborativas.
%
O Noosfero foi desenvolvido na linguagem de programação Ruby, versão 1.8.7, e utiliza
o framework Model-View-Controller (MVC) para aplicações web Ruby on Rails, versão 
2.3.5. A escolha destas tecnologias, por parte dos criadores do Noosfero foi baseada 
fato de que o Ruby possui sintaxe simples, elegante e de fácil leitura, o que aumenta
a manutenibilidade do sistema, uma característica importante num projeto de software
livre que visa atrair desenvolvedores externos~\cite{meirelles2013}.
\section{Desenvolvimento no processo de colaboração ao Noosfero}

\section{Testes no processo de colaboração ao Noosfero}

\section{Funcionalidades desenvolvidas}
