%------------------------------------------------------------------------------%

\chapter{Métodos de Desenvolvimento Empírico}
\label{cap:desenvolvimento-empirico}

O Empirismo baseia-se na aquisição de sabedoria através da percepção do mundo 
externo, ou então do exame da atividade da nossa mente, que abstrai a realidade 
que nosé exterior e as modifica internamente~\cite{chaui2003}.
%
Mecanismos do Controle de Processo Empírico, onde ciclos de feedback constituem o
núcleo da técnica de gerenciamento que são usadas em oposição ao tradicional 
gerenciamento de comando e controle.É uma forma de planejar e gerenciar projetos 
trazendo a autoridade da tomada de decisão a níveis de propriedade de operação e 
certeza~\cite{Schwaber2004}.
%
Neste capítulo será abordado algumas ideias e práticas de processo de desenvolvimento
de software que utilizam métodos ágeis, processos amplamente utilizados no 
desenvolvimentode software livre, outro assunto também abordado no decorrer 
deste texto.
%
\section{Software Livre}

Software livre é uma filosofia que trata programas de computadores como fontes de 
conhecimento que devem ser compartilhados entre a comunidade, evoluindo assim o 
desenvolvimento do pensamento no que se diz respeito a desenvolvimento de software.
%
De acordo com Richard M. Stallman, ativista fundador do movimento software livre, 
um software deve seguir quatro princípios:
%
\begin{enumerate}
\item Liberdade de execução para qualquer uso;
\item Liberdade de estudar o funcionamento de um software e de adaptá-lo às suas 
necessidades
\item Liberdade de redistribuir cópias;
\item Liberdade de melhorar o software e de tornar as modificações públicas de modo 
que a comunidade se beneficie da melhoria~\cite{stallman2001}.
\end{enumerate}
%
Um software é considerado software livre se segue estes quatro princípios, portanto 
o usuário deve poder utilizar, estudar e modificar o software como ele bem entender, 
não significa que o software é necessariamente de graça, mas a partir do momento em 
que se obtém posse de um programa um usuário pode modificar e redistribuir o mesmo 
programa.
%
Para Stallman, a partir do momento em que os custos de desenvolvimento de um software 
são pagos, não há motivos para restrição de acesso, pois a disseminação de conhecimento 
é muito mais benéfica do que potenciais lucros para o produtor.
%

\subsection{GNU e GNU GPL}

Umas das grandes conquistas de Stallman e da Fre Software Foudation (FSF), 
principal organização dedicada a produção e à divulgação do software livre, 
foram o projeto GNU e a licença de software General Public License (GPL).
%
O projeto GNU consistiu em desenvolver um sistema operacional baseado no sistema 
Unix, porém livre de código proprietário, proporcionando aos usuários do Unix um 
sistema totalmente compatível com o Unix, com seu código disponível para todos e 
a liberdade de buscar suporte e personalizações da forma que quisessem.
%
om o GNU, também foi desenvolvida a GPL, licença que dá amparo legal e formaliza 
a ideologia de software livre, amplamente utilizada pelos software livres.
%
No final do desenvolvimento do GNU, o finlandês Linus Torvalds iniciou o 
desenvolvimento de um núcleo de sistema operacional também baseado no Unix e 
deu o nome do núcleo de Linux, disponibilizando-o pela licença GNU GPL. Assim 
foi promovida a integração entre GNU e Linux, criando assim o GNU/Linux, amplamente 
utilizado até os dias de hoje.

%------------------------------------------------------------------------------%

\section{Métodos ágeis}
\label{metodos-ageis}

A utilização de métodos ágeis no desenvolvimento de software tem como caracteristicas 
intrínsecas a flexibilidade e rapidez nas respostas a mudanças. 
%
A agilidade, para uma organização de desenvolvimento de software, é a habilidade 
deadotar e reagir rapidamente e apropriadamente a mudanças no seu ambiente e por 
exigênciasimpostas pelos clientes~\cite{nerur2005}.
%
Os métodos ágeis compartilham valores como comunicação, feedback constante, colaboração 
com o cliente e constante adaptação são baseados no manifesto ágil. Os quatro princípios 
básicos do manifesto ágil mostra o que se espera de qualquer método de desenvolvimento 
desta categoria:
%
\begin{enumerate}
\item Indivíduos  e interações sobre processos e ferramentas;
\item Software funcionando sobre documentação extensiva;
\item Colaboração com o cliente sobre negocioação de contrato;
\item Responder as  mudanças sobre seguir um planejamento;
\end{enumerate}
%
Em projetos ágeis o cliente é mais ativo durante o processo de desenvolvimento, 
determinandoem conjunto com a equipe de desenvolvimento o que será desenvolvido, 
além de participarda validação. Os projetos ágeis também buscam estabelecer um 
tempo determinado e curto para entregade novas releases do sistema, com o objetivo 
de trazer mais satisfação ao cliente.
%
A partir destes curtos ciclos, que são as interações, a equipe de desenvolvimento 
devese preocupar mais com a evolução dos requisitos, que pode gerar mudanças, porém 
mantém o projetoatualizados e diminui o riscos de grandes mudanças a medida que o 
projeto chega ao final.
%
Da perspectiva do produto, métodos ágeis são mais adequados quando os requisitos 
estão emergindo e mudando rapidamente, embora não exista um consenso completo 
neste ponto. De uma perspectiva organizacional, a aplicabilidade pode ser expressa 
examinando três dimensões chaves da organização: cultura, pessoal e comunicação. 
Em relação a estas áreas inúmeros fatores chave do sucesso podem ser identificados~\cite{cohen2004}.

%------------------------------------------------------------------------------%

\subsection{Programação Extrema - XP}

Um método ágil conhecido como Programação extrema, (Extreme Programming - XP) se 
tornou-se bastante popular por utilizar práticas focadas em codificação, como 
programação pareada, integração contínua e desenvolvimento dirigido por testes.
%
O objetivo principal do XP é a excelência no desenvolvimento de software, visando 
um baixo custo, poucos defeitos, alta produtividade e alto retorno de investimento~\cite{sato2007}.
O XP conta com algumas práticas de desenvolvimento para dar suporte à busca pelos 
objetivos citados, essas práticas são: refatoração, integração contínua, testes 
automatizados, código coletivo e programação em pares.

%------------------------------------------------------------------------------%




