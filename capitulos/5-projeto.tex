\chapter{Projeto de Estudo de Caso}

%VERIFICAR AS SECOES DESSE CAPITULO, TIVE DUVIDAS QUANDO COLOQUEI

Este capítulo apresenta a parte de projeto do estudo de caso desenvolvido no trabalho, compondo os objetivos, global e específicos, do estudo, além das questões de pesquisa, questões específicas e métricas que serão utilizadas.

%TODO Definir o que é estudo de caso e basear em um autor para condução do nosso estudo. (Travassos et.al , Basili et al., Wohlin).
%TODO Classificação do estudo: Exploratório, descritivo;

Este estudo de caso é classificado como exploratório, pois não esperamos obter uma resposta definitiva para o problema proposto......

\section{Definição}
Neste trabalho foi apresentado um estudo teórico relacionado à integração das técnicas de usabilidade ao longo do ciclo de vida ágil de desenvolvimento de software livre juntamente com um estudo dos testes de softwares existentes e como estes testes podem impactar a usabilidade de um sistema.

Assim, a proposta deste trabalho consiste na investigação, coleta, análise e discussão dos resultados de dados na adoção de práticas de usabilidade no desenvolvimento de empírico de software.

\subsection{Objetivo Global}

O objetivo global desse estudo é analisar possíveis efeitos da adoção de práticas e técnicas de testes automatizados na usabilidade, durante o processo de desenvolvimento empírico de software.

\subsection{Objetivo de Medição}

Tendo em vista a utilização de várias técnicas de usabilidade e de testes de software em um projeto de desenvolvimento empírico, os objetivos de medição foram definidos à fim de:

\begin{enumerate}
\item Avaliar a influência de práticas de BDD e TDD na usabilidade de um software. 

\item Avaliar o impacto da adoção de técnicas de usabilidade em um contexto específico de desenvolvimento empírico de software.
\end{enumerate}

A partir dos objetivos de medição estabelecidos, foram definidas questões de pesquisa:

\begin{enumerate}
\item Como os testes automatizados são definidos e implementados em um ambiente de desenvolvimento de software empírico?
\item Como inserir os princípios e técnicas de usabilidade dentro do processo ágil de desenvolvimento de software?
\item Como o processo de desenvolvimento utilizando práticas do BDD e TDD podem influenciar os testes de usabilidade?
\end{enumerate}

A partir das quesqtões de pesquisa, foram definidas questões especificas e para cada questão, foram definidas métricas:


%TODO: Definir as questoes especificas a partir das nossas questoes

\begin{enumerate}
\item \textbf{Questão 01: }Qual a estimativa de esforço gasto com testes?

	\textbf{Métrica: } Pontos de história / horas gastas com testes por sprint.

\item \textbf{Questão 02: }Qual a cobertura de testes apresentada?

	\textbf{Métrica: } Porcentagem de cobertura de testes.

\item \textbf{Questão 03: }Qual a viabilidade da adoção de práticas de usabilidade em conjunto com as metodologias ágeis utilizadas no projeto?

	Observar a integração das atividades ao longo do ciclo de vida do projeto e realizar entrevista com os participantes sobre as principais dificuldades encontradas na integração das duas metodologias.
	%isso da questão 3 é uma estrategia de medição e não métrica em si.
\item \textbf{Questão 04: }Qual a conformidade do sistema com as heurísticas de usabilidade?

	\textbf{Métrica: }
\item \textbf{Questão 05: }Qual a capacidade do sistema de atrair o usuário?
	
	\textbf{Métrica: }
\item \textbf{Questao 06:} QUal a capacidade do sistema de possibilitar ao usuário aprender a manuseá-lo?

	\textbf{Métrica: }
\end{enumerate}

%TODO Definir as metricas a partir das questoes especificas


\section{Planejamento}

Nessa seção são apresentados a seleção dos contextos, a formulação das hipóteses, a seleção das variáveis, a seleção dos participantes.

\subsection{Definição de Hipóteses}
%TODO Definir hipóteses e verificar como iremos fazer

\textbf{Hipótese nula (H0)} A integração de práticas de BDD e TDD com práticas de usabilidade no desenvolvimento empírico de software acrescenta melhorias na usabilidade de um software.
%Verificar a palavra 'melhoria', pois é muito ampla

\textbf{Hipótese} A utilização de práticas de BDD/TDD influencia nos testes de usabilidade.

%Obs: Falta definir as hipóteses de forma correta.

\subsection{Seleção do Contexto}

O contexto selecionado para o estudo de caso é o desenvolvimento do novo Portal do Software Público, que vem sendo desenvolvido no LAPPIS a partir das metodologias ágeis Scrum e XP.		

%TODO Qual o objetivo do sistema
%TODO Explicar o ambiente na qual está sendo inserido o estudo (lappis, ministerio, etc)? A linguagem, métodos, ferramentas nas quais estão utilizando.

\subsection{Seleção dos Indivíduos}

%TODO Organizar melhor esse tópico
O projeto do portal do software público envolve a equipe de desenvolvimento do LAPPIS, coordenados pelos professores responsáveis, envolve também a SLTI (Secretaria de Logística e Tecnologia da Informação) e sua direção.

\begin{itemize}

\item Secretária Executiva: responsável pelo financiamento dos projetos do ministério.

\item Diretor: responsável estratégico pelos projetos do ministério, seria o ``Product Owner'' do projeto.

\item Equipe SLTI: responsável pelo levantamento e validação dos requisitos do projeto.

\item Coordenadores: responsáveis estratégicos pela equipe de desenvolvimento do projeto (LAPPIS).

\item Equipe de desenvolvimento: responsável pelo desenvolvimento do portal do software público, que é dividida em duas equipes, uma responsável pelo Colab e outra pelo Noosfero, cada equipe com seu \textbf{coach}.

\end{itemize}

\section{Metodologia}
%TODO Verificar se é esse nome mesmo.

	

\section{Fonte e Método de Coleta de Dados}
%TODO Explicar como será coletado os dados

Os dados foram coletados por meio de entrevistas informais, observações, questionários .....

%Explicar como será realizado os testes de usabilidade

\section{Validade}

As principais ameaças aplicáveis aos estudos de caso são mencionadas por Yin (2010). Dentre elas, estão a validade do constructo, validade interna, validade externa e confiabilidade. %Aline

\section{Estudo de Campo}
%verificar se precisa (TCC Aline)

\section{Cronograma}
\section{Análise}

\section{Considerações finais do capítulo}

Este capítulo apresentou o planejamento do estudo de caso....