\chapter{Projeto de Estudo de Caso}


Este capítulo apresenta a parte de projeto do estudo de caso desenvolvido no trabalho, compondo os objetivos, global e específicos, do estudo, além das questões de pesquisa, questões específicas e métricas que serão utilizadas.



\section{Definição}
Neste trabalho foi apresentado um estudo teórico relacionado à integração das técnicas de usabilidade ao longo do ciclo de vida ágil de desenvolvimento de software livre juntamente com um estudo dos testes de softwares existentes e como estes testes podem impactar a usabilidade de um sistema.


\subsection{Objetivo Global}

O objetivo global desse estudo é analisar possíveis efeitos da adoção de práticas e técnicas de testes automatizados na usabilidade de um sistema, durante o processo de desenvolvimento empírico de software.

\subsection{Objetivo de Medição}

Tendo em vista a utilização de várias técnicas de usabilidade e de testes de software em um projeto de desenvolvimento empírico, os objetivos de medição foram definidos à fim de:

\begin{enumerate}
\item Avaliar a influência de práticas de BDD e TDD na usabilidade de um software. 

\item Avaliar o impacto da adoção de técnicas de usabilidade em um contexto específico de desenvolvimento empírico de software.
\end{enumerate}

\subsection{Questões de Pesquisa}
A partir dos objetivos de medição estabelecidos, foram definidas questões de pesquisa:

\begin{enumerate}
\item Como os testes automatizados são definidos e implementados em um ambiente de desenvolvimento de software empírico?
\item Como inserir os princípios e técnicas de usabilidade dentro do processo ágil de desenvolvimento de software?
\item Como o processo de desenvolvimento utilizando práticas do BDD e TDD podem influenciar os testes de usabilidade?
\end{enumerate}

A partir das quesqtões de pesquisa, foram definidas questões especificas e para cada questão, foram definidas métricas:


%TODO: Definir as questoes especificas a partir das nossas questoes

\begin{enumerate}
\item \textbf{Questão 01: }Qual a estimativa de esforço gasto com testes?

	\textbf{Métrica: } Pontos de história / horas gastas com testes por sprint.

\item \textbf{Questão 02: }Qual a cobertura de testes apresentada?

	\textbf{Métrica: } Porcentagem de cobertura de testes.

\item \textbf{Questão 03: }Qual a viabilidade da adoção de práticas de usabilidade em conjunto com as metodologias ágeis utilizadas no projeto?

	Observar a integração das atividades ao longo do ciclo de vida do projeto e realizar entrevista com os participantes sobre as principais dificuldades encontradas na integração das duas metodologias.
	%isso da questão 3 é uma estrategia de medição e não métrica em si.
\item \textbf{Questão 04: }Qual a conformidade do sistema com as heurísticas de usabilidade?

	\textbf{Métrica: }
\item \textbf{Questão 05: }Qual a capacidade do sistema de atrair o usuário?
	
	\textbf{Métrica: }
\item \textbf{Questao 06:} QUal a capacidade do sistema de possibilitar ao usuário aprender a manuseá-lo?

	\textbf{Métrica: }
\end{enumerate}

%TODO Definir as metricas a partir das questoes especificas


\section{Planejamento}

Nessa seção são apresentados a seleção dos contextos, a formulação das hipóteses, a seleção das variáveis, a seleção dos participantes.

\subsection{Definição de Hipóteses}
%TODO Definir hipóteses e verificar como iremos fazer

\textbf{Hipótese nula (H0)} A integração de práticas de BDD e TDD com práticas de usabilidade no desenvolvimento empírico de software acrescenta melhorias na usabilidade de um software.
%Verificar a palavra 'melhoria', pois é muito ampla

\textbf{Hipótese} A utilização de práticas de BDD/TDD influencia nos testes de usabilidade.

%Obs: Falta definir as hipóteses de forma correta.

\subsection{Seleção do Contexto}

O contexto selecionado para o estudo de caso é o desenvolvimento do novo Portal do Software Público, que vem sendo desenvolvido no LAPPIS a partir das metodologias ágeis Scrum e XP.		

%TODO Qual o objetivo do sistema
%TODO Explicar o ambiente na qual está sendo inserido o estudo (lappis, ministerio, etc)? A linguagem, métodos, ferramentas nas quais estão utilizando.

\subsection{Seleção dos Indivíduos}

%TODO Organizar melhor esse tópico
O projeto do portal do software público envolve a equipe de desenvolvimento do LAPPIS, coordenados pelos professores responsáveis, envolve também a SLTI (Secretaria de Logística e Tecnologia da Informação) e sua direção.

\begin{itemize}

\item Secretária Executiva: responsável pelo financiamento dos projetos do ministério.

\item Diretor: responsável estratégico pelos projetos do ministério, seria o ``Product Owner'' do projeto.

\item Equipe SLTI: responsável pelo levantamento e validação dos requisitos do projeto.

\item Coordenadores: responsáveis estratégicos pela equipe de desenvolvimento do projeto (LAPPIS).

\item Equipe de desenvolvimento: responsável pelo desenvolvimento do portal do software público, que é dividida em duas equipes, uma responsável pelo Colab e outra pelo Noosfero, cada equipe com seu \textbf{coach}.

\end{itemize}

\section{Metodologia}
%TODO Verificar se é esse nome mesmo.
	

\subsection{Guia de usabilidade}
%TODO não sei se aplica nesse item (seria o protocolo)
%Melhorar a escrita

	Propomos  para este estudo de caso o seguinte guia que deve ser executado nas sprints de desenvolvimento do Portal do Software Público:
	
\begin{enumerate}


	\item \textbf{Análise de Usuários}
	
	No ínicio de qualquer projeto é importante conhecer quem são os usuários que irão utilizar o sistema. Para isso existem algumas técnicas para conhecer os usuários.
	%
	A metodologia XPU propõe a criação de Personas e Roteiros para a análise dos usuários, assim como a utilização de questionário de perfil dos usuários e a análise estatísticas de dados.
	
	\begin{itemize}
		\item \textbf{Personas:} Personagens fictícios criados com base em dados reais. As Personas atuam como representantes dos usuários reais e representam as necessidades de um grupo maior. 
		%
		A utilização de Personas permite ter um maior foco no usuário, deixando o projeto centrado no usuário. É utilizado para a identificação de requisitos, criação de cenários e \textit{user stories}. 
		\item \textbf{Roteiros:}		
		
		\item \textbf{Questionários de perfil do usuário} Para identificar o perfil dos usuários do Portal do Software Público é necessário realizar uma pesquisa qualitativa para levantamento das características  dos usuários, de modo a compreender quem são, qual o conhecimento e experiência e como utilizam para realizar seu trabalho acadêmico ou profissional. 
		%
		A análise do questionário servirá para entender o perfil dos usuários do Portal do Software Público, através da investigação de seus interesses. 

		\item \textbf{Dados Estatísticos} Através dos dados estatísticos com ferramentas de análise estatística (\textit{Google analytcs, Piwiki}, entre outros) é possível identificar algumas informações sobre o perfil dos usuários. Esses dados podem ser coletados de base de dados, redes sociais ou sistemas de análises de sites. 

	\end{itemize}
			
		
	\item \textbf{Análise do contexto de uso}

	As Personas e os Roteiros geram as Histórias de usabilidade.

	%\item Definição de Requisitos e Metas de usabilidade

	%Criação de Benchmarks pelos projetistas de interação e usuários para servir como balizadores para avaliar a qualidade da usabilidade que está sendo entregue ao final de cada iteração.
	%Baseado nos atributos de usabilidade são estabelecidos instrumentos de medida para se obter valores quantitativos 	para cada atributo.
	%Esses benchmarks são utilizados para planejar a avaliação de usabilidade e para realizar os testes que irão compor cada avaliação. % esse paragrafo deve ir pro capitulo de usabilidade onde fala de benchmarks.
	
		
	\item \textbf{Planejamento de Usabilidade}

	Estimar os recursos relativos à usabilidade que serão utilizados ao longo do desenvolvimento. Previsões de tempo e custo no planejamento da release. No planejamento da Release do projeto deve-se pensar também em estimativas de tamanho para as histórias de usabilidade e para o planejamento das avaliações de usabilidade.


	\item \textbf{Práticas utilizadas na avaliação da usabilidade}

	Ao longo do ciclo de vida diversas práticas podem ser realizadas para Avaliar a usabilidade do SPB:
	
		
	\begin{itemize}
		\item \textbf{Avaliação Heurística}
		
		É importante fazer uma avaliação heurística para identificar possíveis problemas que possam ser encontrados pelos usuários. Através das Heurísticas de Usabilidade de Nielsen e das listas de verificação.
		
%TODO Ver como sera feito no portal do software público.  %Levantamos algumas tarefas/cenários que devem ser executadas para a realização do teste de usabilidade com o objetivo de medir a satisfação relativa a cada tarefa. As tarefas foram pensadas levando em consideração as necessidades dos usuários do Participa.Br. Tarefas que os usuários-alvo executariam mais frequentemente e tarefas que poderiam apresentar problemas para a compreensão e execução do usuário. 
	
		\item \textbf{Testes com usuários}
		
		No fim de cada sprint de desenvolvimento o usuário deve ser consultado através de um Teste de Usabilidade para coletar o feedback a cerca das funcionalidades desenvolvidas até o momento. E ao final de cada release deverá ser feito um teste de usabilidade com usuários, como é proposto por Nielsen de forma mais detalhada.

		\item \textbf{Protótipos}

		Prototipação é uma prática bastante usada no desenvolvimento de software, que também pode auxiliar na availação de usabilidade.

		\item \textbf{Testes de aceitação - cenários de uso}

		Os cenários de uso dos testes de aceitação descrevem os passos realizados pelos usuários em cada história de usuário, também podem ser utilizados como base para as avaliações de heurísticas e desenvolvimento dos protótipos.

		\item \textbf{Checklists}

		O Ministério do Planejamento possui uma cartilha de usabilidade \footnote{\url{http://epwg.governoeletronico.gov.br}} com checklists que podem ser aplicados em qualquer aplicação web, baseados nesses checklists, criamos um checklist que atende às heurísticas estudadas neste trabalho.

				
	\end{itemize} 
	
	As medidas de usabilidade neste estudo foram obtidas considerando-se três fatores: eficácia, eficiência e satisfação de uso. %TODO (provavlemnet não vamos chegar a medir a satisfação em uso).
	%
	Adotamos como paradigma de avaliação o teste de usabilidade, que consiste em avaliar o desempenho dos usuários na execução de tarefas cuidadosamente preparadas, dentro do escopo do sistema. Esse desempenho pode ser avaliado nos quesitos: número de erros e tempo de execução da tarefa. Com essa avaliação será coletado tantos dados subjetivos em um cenário real como dados objetivos. Os dados subjetivos serão coletados através de opiniões dos participantes, comentários em relação ao uso do portal do software público brasileiro.
	
	O protocolo para o teste de usabilidade seguirá os seguintes passos:
	
	\begin{itemize}

		\item Elabore um cenário de uso com tarefas à serem executadas.
		\item O usuário irá testar o sistema de acordo com o cenário especificado.
		\item Uma pessoa deve observar o que o usuário está fazendo e anotando a sua reação ao utilizar a ferramenta.
		\item Ao final o usuário deve responder algumas perguntas referente as tarefas executadas.

	\end{itemize}

	Elencamos algumas técnicas para avaliar a usabilidade do portal do Software Público:

	\begin{table}[h]
	\begin{tabular}{|l| p{10cm}	|}
		\hline
		Técnica & Descrição \\ \hline
		Observar Usuários & Um observador irá registrar o tempo 
		gasto por cada participante para concluir o estudo de caso, 
		avaliar a ferramenta e se necessitou de alguma ajuda    \\ \hline
		Perguntar aos usuários & Os questionários ASQ e PSSUQ 
		de satisfação dos usuários será utilizado 
		para coletar as opiniões dos participantes.\\ \hline
	\end{tabular}
	\caption{Técnicas de avaliação para os testes com usuários}
	\end{table}
	
	\item \textbf{Controle de Usabilidade}
	
	Ao final de cada interação a velocidade do projeto é avaliada. Deve-se acompanhar as tarefas relativas à usabilidade comparando o esforço previsto com o realizado.
	
\end{enumerate}

Esse protocolo foi extraído do estudo de várias metodologias criadas para a integração das técnicas de usabilidade no contexto de desenvolvimento ágil de software, são elas XPu e XPlus.
 
 
 %TODO Obs: No TCC 1 tinhamos escrito um protocolo de como seria feito á avaliação de usabilidade, algumas coisas podem ser utilizadas e readaptadas para o nosso contexto atual e inserindo as melhorias encontradas por outros métodos.


	

\section{Fonte e Método de Coleta de Dados}
%TODO Explicar como será coletado os dados

Os dados referentes à avaliação das heurísticas de usabilidade foram coletados através de observações e reuniões entre a equipe de desenvolvimento. Quanto aos dados sobre usabildiade do sistema, aplicamos alguns questionários. Em relação aos dados dos testes de aceitação foram coletados pelas rotinas automáticas dos testes.

%Explicar como será realizado os testes de usabilidade

\section{Análise}

Os dados referentes aos questionários e aos testes automáticos seŕão analisados através de gráficos, a fim de servirem de base para a analise dos resultados sobre a usabilidade do sistema.

\section{Considerações finais do capítulo}

Este capítulo apresentou o planejamento do estudo de caso, que será apresentado no próximo capítulo.