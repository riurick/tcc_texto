\chapter{Projeto de Estudo de Caso}

%VERIFICAR AS SECOES DESSE CAPITULO, TIVE DUVIDAS QUANDO COLOQUEI

Este capítulo apresenta a parte de projeto do estudo de caso desenvolvido no trabalho, compondo os objetivos, global e específicos, do estudo, além das questões de pesquisa, questões específicas e métricas que serão utilizadas.

\section{Definição}
Neste trabalho foi apresentado um estudo teórico relacionado à integração das técnicas de usabilidade ao longo do ciclo de vida ágil de desenvolvimento de software livre juntamente um estudo dos testes de softwares existentes e como estes testes podem impactar a usabilidade de um sistema.

Assim, a proposta deste trabalho consiste na investigação, coleta, análise e discussão dos resultados de dados na adoção de práticas de usabilidade no desenvolvimento de software empírico.

\subsection{Objetivo Global}

O objetivo global desse estudo é analisar o impacto da adoção de práticas e técnicas de testes automatizados na usabilidade, durante o processo de desenvolvimento empírico de software.


\subsection{Objetivo de Medição}

Tendo em vista a utilização de várias técnicas de usabilidade e de testes de software em um projeto de desenvolvimento empírico, os objetivos de medição foram definidos à fim de avaliar:

\begin{enumerate}
\item A influência de práticas de BDD e TDD na usabilidade de um software. 

\item Impacto da adoção de técnicas de usabilidade em um contexto específico de desenvolvimento de software.
\end{enumerate}

A partir dos objetivos de medição estabelecidos, foram definidas questões especificas de pesquisa:
%Verificar uma forma melhor de escrever as questões
\begin{enumerate}
%\item A utilização de práticas de usabilidade no ciclo de vida de desenvolvimento de software apresentaram melhores resultados no teste de aceitação?

\item O processo de desenvolvimento utilizando práticas de BDD e TDD apresentam melhores resultados nos testes de usabilidade?
\end{enumerate}
%TODO: As questoes sao redundantes, reescrever. 
%Definir melhor isso
%TODO: Defini as questoes especificas a partir das nossas questoes
%TODO Defini as metricas a partir das questoes especificas

\section{Coleta de Dados}
%TODO Explicar como será coletado os dados

%Explicar como será realizado os testes de usabilidade

\section{Análise}

\section{Considerações finais}