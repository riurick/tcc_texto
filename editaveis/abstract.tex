\begin{resumo}[Abstract]
  \begin{otherlanguage*}{english}  
  
This degree monograph presents a study about software usability and automated testing on the empirical software development. So, the Noosfero plataform (a free software for social networking) was used as case study to apply this study. This firt work presents and applies techniques of usability, in addition to techniques of BDD (Behaviour Driver Development) and TDD (Test Driver Development) in two applications (Portal FGA\footnote{\url{fga.unb.br}} and Comunidade UnB\footnote{\url{comunidade.unb.br}}) of the case study mentioned, thus seeking to ascertain how these testing techniques can affect the assessments of usability of a software. Further work aims to understand how to insert the practical usability studied the life cycle of a free software development (Portal da Participação Social\footnote{\url{participa.br}})  using an agile approach. This study serves as a basis to discussion of some hypotheses abaout empirical development, adressing usability tests. This study also serves as a the basis for the next phase of work, wich will verify the hipotheses about the influence of testing in the usability of a software system.
%TODO: tentei corrigir, mas é melhor vocês reescreverem o abstract e pedir para alguém revisar o inglês...

 
   \vspace{\onelineskip}
 
  \noindent 
  \textbf{Key-words}: free software. automated tests. usability.
  \end{otherlanguage*}
\end{resumo}
