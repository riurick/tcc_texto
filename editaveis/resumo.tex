\begin{resumo}

Este trabalho de conclusão de curso apresenta um estudo sobre usabilidade e testes automatizados no desenvolvimento empírico de software. Para isso, estamos utilizando como estudo de caso a plataforma livre para rede sociais Noosfero. Nessa primeira parte deste trabalho apresentamos como aplicar técnicas de usabilidade, além de técnicas de BDD (desenvolvimento de software dirigido por comportamento) e TDD (desenvolvimento de software dirigido por testes) em dois projetos (Portal FGA\footnote{\url{fga.unb.br}} e Comunidade UnB\footnote{\url{comunidade.unb.br}}) que usam o Noosfero, buscando assim verificar como essas técnicas de testes podem afetar as avaliações de usabilidade de um software.  Além disso o trabalho visa compreender como inserir as práticas de usabilidade estudadas no ciclo de vida de desenvolvimento de um software livre (Portal da Participação Social\footnote{\url{participa.br}}) utilizando uma abordagem ágil. 
Este estudo serviu de base para levantamento de algumas hipóteses a respeito do desenvolvimento empírico, quando pensamos em testes usabilidade. Este estudo também serve como base para a próxima fase do trabalho, que será de verificar as hipóteses levantadas sobre a infulência dos testes automatizados na usabilidade de um sistema de software.


%Done: descrever as siglas BDD e TDD.
%Done: para o TCC 2, o resumo deve explicar e explicitar os objetivos e contribuições do trabalho.

\vspace{\onelineskip}
    
 \noindent
 \textbf{Palavras-chaves}: software livre. testes automatizados. usabilidade.

\end{resumo}
