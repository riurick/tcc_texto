\begin{resumo}

Este trabalho de conclusão de curso apresenta um estudo sobre usabilidade e testes automatizados no desenvolvimento de empírico de software. Para isso, estamos realizando como estudo de caso a plataforma livre para rede sociais Noosfero. Nessa primeira parte deste trabalho apresentamos e estamos investigando como aplicar técnicas de usabilidade, além de técnicas de BDD e TDD em dois projetos que usam o Noosfero, buscando assim verificar como essas técnicas de testes podem afetar as avaliações de usabilidade de um software.  Além disso o trabalho visa compreender como inserir as práticas de usabilidade estudadas no ciclo de vida de desenvolvimento de um software livre utilizando uma abordagem ágil.
%TODO: descrever as siglas BDD e TDD.
%TODO: para o TCC 2, o resumo deve explicar e explicitar os objetivos e contribuições do trabalho.

\vspace{\onelineskip}
    
 \noindent
 \textbf{Palavras-chaves}: software livre. testes automatizados. usabilidade.

\end{resumo}
