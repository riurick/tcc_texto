\begin{resumo}

Este trabalho de conclusão de curso de engenharia de software apresenta um estudo sobre usabilidade e testes automatizados no desenvolvimento empírico de software a fim de verificar como essas técnicas de testes podem se relacionar com as avaliações de usabilidade de um software. Para isso, realizamos um estudo sobre aplicabilidade de técnicas de usabilidade, além de técnicas de BDD (desenvolvimento de software dirigido por comportamento) em projetos de software, além de compreender como inserir as práticas de usabilidade estudadas no desenvolvimento empírico de software. 
Este estudo serviu de base para levantamento e apuração de algumas hipóteses a respeito da relação dos testes automatizados na usabilidade de um sistema de software.


\vspace{\onelineskip}
    
 \noindent
 \textbf{Palavras-chaves}: software livre. testes automatizados. usabilidade.

\end{resumo}
