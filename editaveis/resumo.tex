\begin{resumo}

Este trabalho de conclusão de curso apresenta um estudo sobre usabilidade e testes automatizados de software empírico. Para aplicação deste estudo foi utilizado como estudo de caso a plataforma Noosfero, desenvolvida de forma livre a partir do código aberto. Este estudo apresenta e aplica técnicas de usabilidade, além de técnicas de BDD e TDD em várias aplicações do estudo de caso mencionado, buscando assim verificar como estas técnicas de testes podem afetar as avaliações de usabilidade de um software.  Além disso o trabalho visa compreender como inserir as práticas de usabilidade estudadas no ciclo de vida de desenvolvimento de um software livre utilizando uma abordagem ágil

\vspace{\onelineskip}
    
 \noindent
 \textbf{Palavras-chaves}: software livre. testes automatizados. usabilidade.

\end{resumo}
