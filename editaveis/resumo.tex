\begin{resumo}

Este trabalho de conclusão de curso de engenharia de software apresenta um estudo sobre usabilidade e testes automatizados no desenvolvimento empírico de software a fim de verificar como essas técnicas de tests podem afetar as avaliações de usabilidade de um software. Para isso, estamos utilizando como estudo de caso a plataforma livre para rede sociais Noosfero. Nesta primeira parte, deste trabalho, apresentamos como aplicar técnicas de usabilidade, além de técnicas de BDD (desenvolvimento de software dirigido por comportamento) e TDD (desenvolvimento de software dirigido por testes) em projetos que usam o Noosfero.  Na segunda etapa do trabalho buscamos compreender como inserir as práticas de usabilidade estudadas no ciclo de vida de desenvolvimento de um software livre utilizando uma abordagem ágil. 
Este estudo serviu de base para levantamento de algumas hipóteses a respeito do desenvolvimento empírico de software, quando pensamos em testes usabilidade. Este estudo também serve como base para o trabalho de conclusão de curso 2 deste trabalho, que será de verificar as hipóteses levantadas sobre a influência dos testes automatizados na usabilidade de um sistema de software.


\vspace{\onelineskip}
    
 \noindent
 \textbf{Palavras-chaves}: software livre. testes automatizados. usabilidade.

\end{resumo}
